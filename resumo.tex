Linha de Produto de Software (LPS) é uma metodologia para o desenvolvimento de uma diversidade 
de produtos de software relacionados e sistemas com uso intensivo de software. Durante o desenvolvimento de 
uma LPS, uma ampla variedade de artefatos é criada para ser reusável ao longo 
do desenvolvimento de cada sistema da linha de produto.

Requisitos são um exemplo destes artefatos reusáveis que podem ser instanciados e adaptados para derivar os 
requisitos de produtos específicos. Gerir requisitos em LPS é uma tarefa árdua porque eles são complexos, 
interligados, e divididos em comuns, variáveis e requisitos de um produto específico.  Assim, o processo de 
engenharia de requisitos deve ter suporte ferramental para controlar
a complexidade e o grande volume de requisitos elicitados.

Neste trabalho, propomos uma ferramenta de suporte para realizar a especificação dos requisitos em LPS de forma 
sistemática, através do uso de diretrizes, mostrando passo a passo como a especificação deve ser feita.

\begin{keywords}
linha de produto de software, especificação de requisitos, ferramenta
\end{keywords}