\chapter{Conclusion}
\label{ch:conclusion}

A \acf{SPL} is outlined as a collection of similar software intensive systems
that share a set of common features satisfying the wants of specific customers, market segments 
or mission. Those similar software systems are developed from a set of core assets, comprised of 
documents, specifications, components, and other software artifacts that may be reusable throughout 
the development of each system within the product line
\citep{rafael2013systems}.

Requirements are typical assets in \ac{SPL}. They are specified in reusable models,
in which commonalities and variabilities are documented explicitly. Thus, these 
requirements can be instantiated and adapted to derive the requirements for an 
individual product \citep{cheng2007research}. New products in the SPL will be
much simpler to specify, because the requirements are reused and tailored
\citep{clements2002software}.

\acf{RE} in \ac{SPL} has an additional cost. Many \ac{SPL} requirements are
complex, interlinked, and divided into common, variable and product-specific requirements 
\citep{birk2003report,de2014defining}.  The requirements engineering process
must be tool-supported to handle complexity and the huge volume of elicited requirements
\citep{birk2003report}.

In this work, we proposed a support tool for performing the
specification of the \ac{SPL} requirements in a systematic way through the use of guidelines,  
showing step by step how the specification should be done. In addition, an
initial evaluation was performed in order to point out negative and positive
points of the tool and direct us to future improvements to be done.

\section{Research Contribution}
\label{sc:reserach-contrib}
As a result of the work presented in this dissertation, the following contribution can be highlighted:
\begin{itemize}
\item \textbf{Tool support for a SPL domain requirements specification approach
(FeDRE)} 
We extended the tool \ac{SPLICE}, a \ac{SPL} lifecycle management tool
and automated \acf{FeDRE}, thus improving the automation of Software Product Lines (\ac{SPL}) 
requirements engineering phase.
\end{itemize}

\section{Future Work}
\label{sc:future-work}
An initial version of the tool was developed and evaluated in this work.
However, we are aware that some enhancements and features must be implemented.
Also, some defects must be fixed. 
In this work, we presented a survey. A more
detailed evaluation is needed, for example, a controlled experiment or a case
study with a higher number of respondents in order to provide richer findings for the stakeholders.

