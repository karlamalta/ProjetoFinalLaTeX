\chapter{Conclusion}
\label{ch:conclusion}

\begin{quotation}[The End of the Beginning]{Winston Churchill }
Now this is not the end. It is not even the beginning of the end. But it is, perhaps, the end of the beginning.
\end{quotation}

\acf{SPL} has proven to be the methodology for developing a diversity of software products and software-intensive systems at lower costs, in shorter time, and with higher quality. Many reports document the significant achievements and experience gained by introducing software product lines in the software industry \citep{Pohl2005}. However, the complexity of product lines processes implicate that tool support is inevitable to facilitate smooth performance and to avoid costly errors \citep{Dhungana2007}. 

An upcoming concept is the \acf{ALM}, which deals with approaches, methodologies and tools for integrated management of all aspects of software development. Its goal is to making software development and delivery more efficient and predictable by providing a highly integrated platform for managing the various activities of the development lifecycle from inception through deployment \citep{kaariainen2011towards}. Chapter 2 summarized the basic concepts about software product lines, \ac{ALM} tools, \ac{CASE} tools and their aspects. We also presented an informal study of features available in commercial tools.

In this sense, in order to facilitate the process usage and to aid the \acf{SE} during the \ac{SPL} process, this dissertation presented, in Chapter 3, a lightweight metamodel for \ac{SPL} using agile methodologies. We also desbribed in details, the \acf{SPLICE} tool, build in order to support and integrate \ac{SPL} activities, such as, requirements management, architecture, coding, testing, tracking, and release management, providing process automation and traceability across the process. We presented the functional and non-functional requirements, its architecture, as well as theinvolved frameworks and technologies.

An evaluation of the \ac{SPLICE} tool was presented in Chapter 4, with a case study conducted inside the research laboratory. The results demonstrated that the \ac{SPLICE} address the traceability problem and was considered useful to all experts.

\section{Research Contribution}
\label{sc:reserach-contrib}
The main contributions of this research are described as follows:
\begin{itemize}
\item  \textbf{Informal review of available \ac{ALM} tools}. Through this study, was conducted an informal search for similar commercial or academic tools. Fourteen characteristics and functionalities that is interesting to be covered in an Application lifecycle tool.  A table was created comparing all the tools, that companies can use as a way to identify the best tool according to their needs.

\item  \textbf{ SPL Lightweight Metamodel}. Based on a previous metamodel, and the need for applying agile methodologies, we introduced a proposal for a lightweight metamodel that represents the interaction among asserts of a \ac{SPL} during its lifecycle.
\item  \textbf{ SPLICE tool}. To implement the metamodel proposed, it was presented a web-based tool for \ac{SPL} lifecycle management, including the set of functional and non-functional requirements, architecture, frameworks and technologies adopted during its construction.
\item  \textbf{Case Study}. After the tool development, a case study was performed to evaluate the \ac{SPLICE} tool and gather opinions and critics about the product, to guide further development.

\end{itemize}

\section{Future Work}
\label{sc:future-work}
An initial prototype was developed and evaluated in this work. However, we are aware that some enhancements and features must be implemented, as well as some defects must be fixed. Experts reported some of the enhancements and defects during our survey, and others were left out because they were out of scope of a graduation project. Thus, some important aspects are described as follows:

\begin{itemize}

\item  \textbf{ Visual metamodel editing}. One main objective of this work is to present a flexible tool, giving the possibility to adapt the implemented metamodel for specific project needs. Therefore, a visual editor must be implemented during the project setup to be more intuitive to users.

\item  \textbf{ Source code variation.} The tool actually do not support source-code variability, since the users are only able to perform assets derivation. With source-code variation support, we could archive a complete derivation process.

\item  \textbf{ Risk Management}. According to \cite{Sommerville2011}, Risk Management (RM) is important because of the inherent uncertainties that most projects face. We did not implement this model in our metamodel. As consequence, we intend to consider this feature in the next improvements. 

  
\item  \textbf{ Further evaluation}. In this work, we presented a case study. A more detailed evaluation is needed by applying the proposed case study protocol in other contexts in order to provide richer findings for the stakeholders. 
\end{itemize}
