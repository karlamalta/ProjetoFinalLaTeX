\chapter{SPLICE-FeDRE: a SPL Domain Requirements Specification Tool}
\label{ch:splice}

\section{Introduction}

In this chapter, we present functional and non-functional requirements for a tool we call SPLICE-FeDRE, 
and its implementation. The tool is an extension of \acf{SPLICE}, built in order
to support and integrate \ac{SPL} activities, such as, requirements management,
architecture, coding, testing, tracking, and release management, providing process automation and 
traceability across the process.

The remainder of this chapter is organized as follows: \secref{sc:fedre}
presents the \ac{FeDRE} approach; \secref{sc:splice} describes the tool
\ac{SPLICE}; \secref{sc:requirements} presents the requirements of SPLICE-FeDRE;
details of the implementation of SPLICE-FeDRE are discussed in
\secref{sc:implementation}; \secref{sc:operation} shows the tool SPLICE-FeDRE
in operation; and, finally, \secref{sc:solutionsummary} presents the summary of
the chapter.
 
\section{FeDRE Overview}
\label{sc:fedre}

The \acf{FeDRE} approach \citep{de2014defining} for 
\ac{SPL} has been defined by considering the feature model as the main artifact for specifying \ac{SPL} requirements. 
The aim of the approach is to perform the requirements specification by systematically utilizing the features 
identified in the \ac{SPL} domain through the use of guidelines that establish traceability links between features 
and requirements.

The main activities of the \ac{FeDRE} approach are: Scoping and Requirements Specification for Domain Engineering. 
\figref{fg:fedre-activities} shows the activities of \ac{FeDRE}, which are
detailed in this chapter. The following roles are involved in these activities: Domain Analyst, Domain Expert, Market Expert and the Domain Requirements Analyst.

\begin{figure}[htp]
\begin{center}
  \includegraphics[width=11cm]{chapters/proposed_solution/img/captures/fedre_activities.png}
  \caption[Overview of the FeDRE approach]{Overview of the FeDRE approach \citep{de2014defining}}
  \label{fg:fedre-activities}
\end{center}
\end{figure}

\subsection{Scoping}

The first activity performed in \ac{FeDRE} is the Scoping. This determines not only
what products to include in an \ac{SPL} but also whether or not an organization should 
launch the \ac{SPL}. Three main artifacts are produced as a result of the Scoping activity: 
the Feature Model, the Feature Specification, and the Product Map, using the Existing 
Assets (if any) as the input artifact. These three artifacts will drive the \ac{SPL} requirements 
specification for domain engineering. Each of these artifacts (input and outputs) is 
detailed below.

\subsubsection{Existing Assets}

Existing assets (e.g., user manual or existing systems) help the Domain Analyst
and the Domain Expert to identify the features and products in the \ac{SPL}. When they do not 
exist, a proactive approach can be followed to build the \ac{SPL} from scratch.

\subsubsection{Feature Model}

Feature modeling is a technique that is used to model common and variable
properties, and can be used to capture, organize and visualize features in the \ac{SPL}.

\subsubsection{Feature Specification}

The Domain Analyst is responsible for specifying the features using a feature
specification template. This template captures the detailed information of the
features and maintains traceability with all the artifacts involved.

\subsubsection{Product Map}

Each of the identified features is assigned to the corresponding products in the \ac{SPL}. 
The set of relationships among features and products produces the Product Map artifact, 
which describes all the features that are required to build a specific product in the \ac{SPL}.
All these artifacts are the input for the Requirements Specification for Domain Engineering 
activity, which is described next.

\subsection{Requirements Specification for Domain Engineering}

This activity specifies the \ac{SPL} requirements for domain engineering. These requirements 
allow realization of the features and desired products identified in the Scoping activity. 
The steps required to perform this activity are described in the Guidelines for Specifying 
SPL Functional Requirements, Sub-section \ref{subsec:guidelines} below.

This activity, seen in \figref{fg:fedre-activities}, uses the Feature Model,
Feature Specification and Product Map as input artifacts and produces the Glossary, Functional Requirements and Traceability Matrix 
as output artifacts. Each of these output artifacts is detailed below.

\subsubsection{Glossary}

The Glossary describes and explains the main terms in the domain in order to
provide the stakeholders with a common vocabulary and avoid misconceptions.

\subsubsection{Functional Requirements}

This artifact contains all the functional requirements identified (common or
variable), for the family of products that constitute the \ac{SPL}. Use cases are used 
to specify the \ac{SPL} functional requirements. Each functional requirement has a unique 
Use case id, a Name, a Description, Associated Feature(s), Pre and Post-Conditions, and 
the Main Success Scenario. A functional requirement can also be related to an Actor and 
may have Include and/or Extend relationships with other use case(s).

\subsubsection{Traceability Matrix}

The Traceability Matrix is a matrix that contains the links among features and
the functional requirements.

\subsection{Guidelines for Specifying SPL Functional Requirements}
\label{subsec:guidelines}

The purpose of the guidelines is to guide the Requirements Analyst in 
the specification of \ac{SPL} functional requirements for domain engineering. The 
guidelines have been structured to specify functional requirements by addressing the following 
questions: i) Which features or set of features will be grouped to be specified by use 
cases? ii) What are the specific use cases for the feature or set of features? iii) Where 
should the use cases be specified? (when there is a set of features in a hierarchy, do we specify 
the use cases for each individual feature or only for the parent features?) and iv) How is the use 
case specified in terms of steps?

Activities, tasks and steps are used in the process of specifying requirements
for \ac{SPL}. \figref{fg:fedre-guide} shows the guidelines with the detailed
steps of each task for specifying \ac{SPL} functional requirements.

\begin{figure}[htp]
\begin{center}
  \includegraphics[width=11cm]{chapters/proposed_solution/img/captures/fedre_guide.png}
  \caption[Guidelines For Specifying SPL Functional Requirements]{Guidelines For Specifying SPL Functional Requirements \citep{de2014defining}}
  \label{fg:fedre-guide}
\end{center}
\end{figure}

\section{SPLICE Overview}
\label{sc:splice}

\acf{SPLICE} \citep{splice2014cbsof}  is an open source (GNU General Public
License), Python,  web-based software product line lifecycle management tool, providing 
traceability and variability management and supporting most of the \ac{SPL} process 
activities such as scoping, requirements, architecture, testing, version control, 
evolution, management and agile practices. This tool assists the engineers involved 
in the process, with the assets creation and maintenance, while providing traceability 
and variability management, as well offering detailed reports and enabling engineers 
to easily navigate between the assets using the traceability links.

\subsection{Metamodel}

\ac{SPLICE} proposes a lightweight metamodel, representing the interactions among \ac{SPL}
assets, developed in order to provide a way of managing traceability and variability. 
The proposed metamodel represents the reusable assets involved in a \ac{SPL} project, and 
simplified description of the models is presented next.

\begin{itemize}
\item \textbf{Scoping Module} comprises the Feature and the Product Model. Many artifacts relates 
directly with the Feature Model including Use Case, Glossary, User Story and Scope Backlog. 
A Product is composed of one or more Features.

\item \textbf{Requirements Module} involves the requirements engineering
traceability and interactions issues, considering the variability and commonality in the \ac{SPL} 
products. The main object of this \ac{SPL} phase is Use Case. The concept of User stories is used 
in this metamodel to represent what a user does or needs to do as part of his or her job function. 
 
\item \textbf{Testing Module} is composed of a name, description, the
Expected result and a set of Test Steps. One Test Case can have many Test Execution that 
represent one execution of it. The reasoning for the Test Execution is to enable a test 
automation machinery. The metamodel also represents the acceptance testing with the Acceptance 
Test and Acceptance Test Execution. 

\item \textbf{Agile Planning Module} contains Sprint Planning models, which are composed of a number of Tickets, a
deadline, an objective and a start date. At the end of the sprint, it happens a retrospective, 
represented in the model by Sprint Retrospective, that contains a set of Strong Points and Should be 
Improved models that express what points in the spring was adequate, and what needs improvement.
\end{itemize}

\subsection{Main Functionalities}

The main functionalities of \ac{SPLICE} include:

\begin{itemize}
\item  \textbf{Metamodel Implementation.} All the screens are completely
auto-generated based on the models descriptions, allowing the Software Engineer to 
easily modify the process. For every model, a complete \acf{CRUD} system is
created. The \ac{SPLICE} also provides advanced features such as filtering and classification.
\item  \textbf{Issue Tracking.} \ac{SPLICE} has a full-featured Issue Tracking. It
was extended to implement \ac{SPL} specific features and to provide traceability between other assets.
\item  \textbf{Traceability.} \ac{SPLICE} provides total traceability for all assets
in the metamodel, and is able to report direct and indirect relations between them. In reports, assets 
have hyperlinks, enabling the navigation between them.
\item  \textbf{Custom SPL Widgets.} \ac{SPLICE} has a set of custom widgets to
represent specific \ac{SPL} models. Such as Feature Map, Product Map, and Agile Poker planning.
\item  \textbf{Change history and Timeline.} \ac{SPLICE} has a rich set of features
to visualize how the project is going, where the changes are happening, and who did it. For every 
Issue or Asset, a complete Change history is recorded.
\item  \textbf{Unified Control Panel.} The tool aggregates the configuration of
all external tools in a unified interface. With the same credentials, the user is able to access all 
\ac{SPLICE} features, including external tools as \acf{VCS}.
\item  \textbf{Agile Planning.} The \ac{SPLICE} supports a set of Agile practices
such as effort estimation, where team members use effort and degree of difficulty to estimate 
their own work. The Features can be dragged by the mouse, and their position is updated in accordance.
\item  \textbf{Automatic reports generation.} \ac{SPLICE} has the ability of creating
reports, including PDFs. The generated report includes a cover, a summary and the set of 
the chosen artifact related to the product. This format is suitable for the requirements 
validation by stakeholders. The tool is also able to collect all reports for a given Product, and 
create a compressed file containing the set of generated reports.
\end{itemize}

\section{SPLICE-FeDRE Requirements}
\label{sc:requirements}
In the SPLICE-FeDRE specification, the following functional requirements were
defined:

\begin{itemize}
\item  \textbf{FR1 - FeDRE Feature Specification.} The tool should provide a complete 
\ac{CRUD} (Create, Read, Update and Delete) for the model Feature that satisfies the \ac{FeDRE} 
approach needs. The model Feature should include a unique Feature id, name, description, 
priority (high, medium or low), type (abstract or concrete), variability (mandatory, optional, OR ou XOR), 
binding time (compile, runtime),  parent feature, glossary, use case diagram, similar feature(s), 
required feature(s) and excluded feature(s).

\item  \textbf{FR2 - FeDRE Use Case Specification.} The tool should provide a
complete \ac{CRUD} (Create, Read, Update and Delete) for the model Use Case that satisfies the 
\ac{FeDRE} approach needs. The model Use Case should include a unique Use case id,  name,  description, 
associated feature(s), pre-conditions, post-conditions, and the main success scenario. A Use Case can 
also be related to an actor and may have include and/or extend relationships with other use case(s), 
and alternative scenarios.

\item  \textbf{FR3 - FeDRE Guidelines.} The tool shoul implement the \ac{FeDRE}
guidelines for specifying \ac{SPL} functional requirements.  The Features must be stored  
hierarchically in order to enable \ac{FeDRE} guidelines implementation. This  should be done by 
storing Features as an n-ary tree data structure that  represents the Feature Model. 

\end{itemize}

\section{SPLICE-FeDRE Implementation}
\label{sc:implementation}

The tool \ac{SPLICE} was implemented using the Django Framework and the Python programming language. 
According to \citep{python2014}, “Python is a dynamic object-oriented
programming language that is used in a wide variety of application domains. It has a very clear, readable syntax, 
offers strong support for integration with other languages and tools, comes with extensive standard 
libraries, and can be learned in a few days. Many Python programmers report substantial productivity 
gains and feel the language encourages the development of higher quality, more maintainable code”. 
Other languages used to developed the tool were JavaScript, CSS, HTML, XML, YAML and make.

According to the \ac{SPLICE} developer, this choice was motivated by the unprecedented flexibility that 
the Django \acf{ORM} empowers it users, making the metamodel changes
effortless.
Also, Python is becoming the introductory language for a number of computer science curriculums  
\citep{Sanders2008}. Python is also frequently used on many scientific workflows
\citep{Bui2010} making the project attractive for future data scientists
experiments and for undergraduate projects.
 
The implementation of the new version called SPLICE-FeDRE adopted the same set of programming languages and 
framework for development. In SPLICE-FeDRE, the features are stored hierarchically using a modified preorder 
tree traversal algorithm. We can think of the feature model as 
an n-ary tree of features, where the root node is a special node that represents de product line. This 
tree is traversed using a depth-first search algorithm, then the \ac{FeDRE} flow of activities, tasks and 
steps is executed for each subtree of the root node, one at a time. 

\section{SPLICE-FeDRE in action}
\label{sc:operation}

In order to demonstrate how the tool SPLICE-FeDRE works, this section shows the 
operation of selected features, with a brief description.

\subsection{Feature Specification}

In the home page of SPLICE-FeDRE, seen in \figref{fg:assets}, an assets menu can
be used to manage the assets available.

\begin{figure}[htp]
\begin{center}
  \includegraphics[width=14cm]{chapters/proposed_solution/img/captures/assets.PNG}
  \caption[Assets screen]{Assets screen}
  \label{fg:assets}
\end{center}
\end{figure}

By clicking in Features, an user can have access to a complete \acf{CRUD}, 
which is also available for all the models listed in the assets menu. \figref{fg:add-feature} 
shows part of the form used to add a new Feature while specifying the Features. It implements the functional
requirement \textit{FR1- Feature Specification}. The model Feature includes a unique Feature id, 
name, description, priority (high, medium or low), type (abstract or concrete), variability 
(mandatory, optional, OR ou XOR), binding time (compile, runtime), parent feature, glossary, use case 
diagram, similar feature(s), required feature(s) and excluded feature(s). 

\begin{figure}[htp]
\begin{center}
  \includegraphics[width=14cm]{chapters/proposed_solution/img/captures/add_feature.PNG}
  \caption[Add feature form]{Add feature form}
  \label{fg:add-feature}
\end{center}
\end{figure}

\subsection{FeDRE Guidelines}

The \ac{FeDRE} page is depicted in \figref{fg:start-fedre}. Once the features
specification is finished, the user can start the flow of \ac{FeDRE} guidelines in order to specify the requirements of each subtree of features, 
one by one. It implements the functional requirement \textit{RF3 – FeDRE
Guidelines}.

\begin{figure}[htp]
\begin{center}
  \includegraphics[width=14cm]{chapters/proposed_solution/img/captures/start_fedre.PNG}
  \caption[FeDRE initial screen]{FeDRE initial screen}
  \label{fg:start-fedre}
\end{center}
\end{figure}

For each branch, the user can see a hierarchy of this branch features, and lists of the features that 
must have use cases, the features that may have use cases and the features that should not have use cases. 
Also, it is shown a list of steps to be accomplished before moving to the next
branch, as seen in \figref{fg:branch1}.

\begin{figure}[htp]
\begin{center}
  \includegraphics[width=14cm]{chapters/proposed_solution/img/captures/branch1.PNG}
  \caption[Branch example]{Branch example}
  \label{fg:branch1}
\end{center}
\end{figure}

\subsection{Use Case Specification}

As mentioned above, a complete \ac{CRUD} is also accessible for the model Use
Case.
Implementing the requirement \textit{FR2 - Use Case Specification}, the model
Use Case includes a unique Use case id,  name,  description, associated feature(s), pre-conditions, 
post-conditions, 
and the main success scenario. A Use Case can also be related to an actor and may have include and/or 
extend relationships with other use case(s), and alternative scenarios. See
\figref{fg:add-use-case} below:

\begin{figure}[htp]
\begin{center}
  \includegraphics[width=14cm]{chapters/proposed_solution/img/captures/add_use_case.PNG}
  \caption[Add use case form]{Add use case form}
  \label{fg:add-use-case}
\end{center}
\end{figure}

\section{Summary}
\label{sc:solutionsummary}

In this chapter, it was presented the \acf{FeDRE} approach and the \acf{SPLICE},
a web-based tool for \ac{SPL} lifecycle management, and how it was extended to 
automate the \ac{FeDRE} approach. Next chapter presents an evaluation of SPLICE-FeDRE 
performed during the development of the tool. 


