\chapter{SPLICE-FeDRE Evaluation}
\label{ch:caseStudy}

\section{Introduction}
\label{sc:experimentIntroduction}

This chapter describes a survey applied to validate the tool developed for this
work. It is organized as follows: \secref{sc:definition} defines this evaluation; in \secref{sc:researchMethod} the 
data collection model is presented; \secref{sc:datacollection} describes the results and its interpretation; the 
\secref{sc:threats} analyzes the threats to validity of the evaluation; \secref{sc:leassonsLearned} and \secref{sc:expsummary} describe some 
findings and summarize the chapter.

\section{Definition}
\label{sc:definition}

\subsection{Context}
A survey was applied, after the complete implementation of SPLICE-FeDRE, in
order to validate the application developed in regard to its usefulness when it 
comes to handling complexity and scalability problems during requirements specification. 
The survey was applied at Software Engineering Laboratory, in November 2015, and we had 
two participants that were master students, all members of \acf{RiSE} Lab.
\subsection{Research Questions}

In this evaluation the main objective is to analyze the usefulness of the tool for handling 
complexity and scalability problems that may arise during requirements specification phase 
of a \acf{SPL}. In order to evaluate these aspects in our proposal, we 
defined two research questions:
\begin{itemize}
\item \textbf{Is the proposed tool useful for handling compexity during the SPL Requirements Engineering process?}

Rationale: The goal is to verify if the tool helps handling complexity during the \ac{SPL} 
Requirements Engineering process.
\item \textbf{Is the proposed tool useful for handling scalability problems during the SPL Requirements Engineering process?}

Rationale: The goal is to verify if the tool helps handling scalability problems during the \ac{SPL} 
Requirements Engineering process.

\end{itemize}

\section{Data collection}
\label{sc:researchMethod}

\subsection{Survey Design}
The data collection instrument selected in this evaluation is the Expert Survey. A survey is a 
mechanism of data gathering in which participants answers questions or statements previously 
developed and they are probably the most commonly used instrument to gather opinions from experts 
according to \citep{Kitchenham2008}.

This survey design is based on the design proposed by \citep{Kitchenham2008} and it is 
composed of a set of personal questions, closed-ended and open-ended questions related to the research 
questions. Also, to give the respondents the necessary understanding about the application, a training 
was offered to them. The remain of this section contains the overall process applied in this evaluation 
and the methodology used. 
\subsection{Developing the Survey Instrument}
The questionnaire was composed of three personal questions, five closed
questions with justification fields, and three open questions. The closed questions were 
formulated to measure and quality the data, while getting personal feedback. The open questions 
were built to collect the experts’ experiences and their impressions about the tool. The questionnaire 
can be seen in the Appendix 1.

\subsection{Analyzing the data}
In order to collect the data, the experts filled a printed questionnaire. After designing and running 
the survey, the next step was to analyze the collected data. The main analysis procedure was to check 
all responses, tabulate the data, identify the findings and classify the options.

\section{Results}
\label{sc:datacollection}
In this section the analysis of the collected data are presented, discussing the
given answer for each question.

\subsubsection{Respondents experience}
The first three questions were personal questions such as name and experience,
and their answers are summarized in Table \ref{table:expertsselected}.

\begin{table}
\begin{center}
\centering
\small
\tabcolsep=0.11cm
    \begin{tabular}{|l|l|l|l|}
    \hline
    Respondent & Occupation & RE experience & SPL experience           
    \\ \hline 1 & M.Sc student & 4 years & 3 years and 5 months
    \\ \hline 2 & M.Sc student & 6 years & 3 years 
    \\ \hline
    \end{tabular}
        \caption {Experts Selected}
        \label{table:expertsselected}
        \end{center}

\end{table}

\subsubsection {Tool usage difficulties}
Considering the questions \textbf{“Did you have any difficulty during the
execution of any activity in the tool?”} and  \textbf{“Did you have any problems
creating  use cases?”}, none of the respondents reported any difficulty to use
the tool.

\subsubsection{Tool Helpfulness}
In the question \textbf{“Do you think that the proposed tool would aid you
during a SPL Requirements Engineering process? Would you spontaneously use the
tool hereafter?”} all the respondents agreed that the tool would aid them to specify 
requirements and would keep using the tool 
from that moment on. However, one of them stated that the \ac{FeDRE} steps could
be more detailed in the tool.


Considering the question \textbf{“Do you think the proposed tool is useful to
handle the complexity of SPL Requirements Engineering process?”}, both
respondents answered yes, and one of them stated that the tool reduces the effort to specify 
requirements using \ac{FeDRE} guidelines.


All the respondents indicated in the question \textbf{“Do you think the proposed
tool is useful to handle scalability problems during a SPL Requirements
Engineering process?”} that the proposed tool is useful to handle scalability problems that may arise during  
\ac{SPL} Requirements Engineering process.

\subsubsection{Positive points}
We asked the experts the question \textbf{What are the positive points of using the tool?}, and the positive 
points of the tool according to them are:
\begin{itemize}
\item Centralized information;
\item More accuracy when choosing the use cases to be specified;;
\item Time saving.
\end{itemize}

\subsubsection{Negative points}
In Contrast with the previous question, we also asked \textbf{What are the
negative points of using the tool?}.
Only one point was mentioned, as follow:
\begin{itemize}
\item An external tool is needed to draw the use case diagrams.
\end{itemize}

\subsubsection{Suggestions}
Lastly, we asked \textbf{“Please, write down any suggestion you think might be
useful”}. One expert suggested that we implement a use case drawing feature as part of the tool 
to avoid the use of external tools. The another expert suggested a more illustrated interface.

\section{Threats to validity}
\label{sc:threats}
There are some threats to the validity of our study, which were briefly
described and discussed:
\begin{itemize}
\item \textbf{Research questions.} The research questions we defined cannot
provide complete coverage of all the features covered by the tool. We considered just some important points: complexity and 
scalability problems handling.

\item \textbf{Sample size.} The number of respondents is an important detail in
a survey. Due to the limited availability of respondents with a \ac{SPL} background, the evaluation
may contain biases.
A higher number of participants helps  generalizing the results obtained.

\item \textbf{Quality of training.} The quality of the training conducted before
applying the questionnaire may have compromised the correct understanding of the \ac{FeDRE} approach and 
the application evaluated.

\item \textbf{Translation of the answers.} All the responses were written in
portuguese and translated to english by the author. This may have changed the direction of the response.

\end{itemize}

\section{Findings}
\label{sc:leassonsLearned}
Analyzing the answers, none of the respondents reported difficulties during the tool usage. No major 
usability problem was found, and all of them were able to use and evaluate the tool without supervision. 
However, one of them suggested a more illustrative and interactive user interface.
 
All the experts explicitly declared that the tool was useful to handle complexity and scalability problems 
during \ac{SPL} requirements specification. They also stated that would, spontaneously, use the tool in future \ac{SPL} 
projects.

The positive points of the tool, in their opinion, are that all the needed information is concentrated in the 
tool. Also, the tool decides automatically what features should and should not generate use cases, and that 
automation saves their time and promotes more accuracy in the process, thus avoiding human mistakes. 

The negative point pointed out by one of them is that an external tool must be used to draw use case diagrams. 
Ideally, the tool should offer this feature to ease the process.
\section{Summary}
\label{sc:expsummary}

This chapter presented the definition, planning, analysis and interpretation of a survey 
to evaluate the SPLICE-FeDRE tool. The survey was applied to experts at Software Engineering Laboratory. 
The two participants were members of \ac{RiSE} Lab. After concluding the  
questionnaires, we gathered information that can be used as a guide to improve the tool, and an indicator 
about the actual status of the tool. 
The results of the experiment pointed out that the SPLICE-FeDRE addresses the 
complexity and scalability problems that may arise during \ac{SPL} requirements
engineering phase.
However, some points of improvements were raised, that we plan to fix on future versions. Next 
chapter presents the concluding remarks and future work of this dissertation.


