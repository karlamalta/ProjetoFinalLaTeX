\chapter{Introduction}
\label{ch:introduction} 
A \acf{SPL} is outlined as a collection of similar software intensive systems
that share a set of common features satisfying the wants of specific customers, market segments 
or mission. Those similar software systems are developed from a set of core assets, comprised of 
documents, specifications, components, and other software artifacts that may be reusable throughout 
the development of each system within the product line
\citep{rafael2013systems}.

Requirements are typical assets in \ac{SPL}. They are specified in reusable models,
in which commonalities and variabilities are documented explicitly. Thus, these 
requirements can be instantiated and adapted to derive the requirements for an 
individual product \citep{cheng2007research}. New products in the SPL will be
much simpler to specify, because the requirements are reused and tailored
\citep{clements2002software}.

\acf{RE} in \ac{SPL} has an additional cost. Many \ac{SPL} requirements are
complex, interlinked, and divided into common, variable and product-specific requirements 
\citep{birk2003report,de2014defining}.  The requirements engineering process
must be tool-supported to handle complexity and the huge volume of elicited requirements
\citep{birk2003report}.

The focus of this dissertation is to provide a support tool for performing the specification of the 
\ac{SPL} requirements in a systematic way through the use of guidelines,  showing step by step how the 
specification should be done.

This chapter contextualizes the focus of this dissertation and starts by
presenting its motivation in \secref{sc:motivation} and a clear definition of the problem in 
\secref{sc:problem}. A brief overview of the proposed solution is presented in
\secref{sc:related}, while \secref{sc:outofscope} describes some aspects that
are not directly addressed by this work.
\secref{sc:contributions} presents the main contributions,  
\secref{sc:design} presents the research design  and, finally,
\secref{sc:structure} outlines the structure of this dissertation.

\section{Motivation}
\label{sc:motivation}
Within the \ac{SPL} paradigm, it is very important to perform a good requirements
engineering phase, because it is the basis of the  \ac{SPL} paradigm. However, existing 
tools are not designed to support the requirements engineering process for software 
product lines. Existing tools support only single product development and therefore 
lack support for modeling commonalities and variabilities as well as variation points in 
requirements \citep{birk2003report}.

Some approaches have been proposed to perform the specification and evolution of
the \ac{SPL} requirements in a systematic way through the use of guidelines: 
\acf{FeDRE} and \acf{FeDRE2}.
These approaches are considered easy to use and useful, however, they do not have a support tool. 
The lack of tool support can lead to mistakes during the manual execution of the guidelines, moreover, 
without a tool support these approaches can have problems with scalability.

In this sense, a \ac{SPL} Requirements Engineering tool is proposed to automatize the
\ac{SPL} requirements specification activities according to the \ac{FeDRE} approach. This tool is 
an extension of the tool \ac{SPLICE} \citep{splice2014cbsof}, which is an
integrated tool for developing \ac{SPL}.

\section{Problem Statement}
\label{sc:problem}

This work investigates the problems of complexity and scalability in \ac{SPL}
requirements specification phase to understand its activities in order to improve 
automation of these activities. This work promotes effort and mistakes reduction during 
\ac{SPL} requirements specification by poviding a \ac{SPL} Requirements Engineering tool .   

\section{Related Work}
\label{sc:related}
\acf{FeDRE} \citep{de2014defining} was 
defined and evaluated to aid developers in the \acf{RE} activity for \ac{SPL} 
development. The \ac{FeDRE} focus is the requirements specification in the Domain Engineering activity. 
\ac{FeDRE} realizes chunks of features from a feature model into functional requirements, which are then 
specified by use cases. Also, it provides detailed guidelines on how to specify the requirements.  
A first evaluation of \ac{FeDRE} was performed through an empirical study within a \ac{SPL} project, where \ac{FeDRE} 
was perceived as easy to learn and useful by the participants.

\acf{SPLICE} is a web-based \ac{SPL} life-cycle 
management tool that provides traceability and variability management and supports most of the \ac{SPL} process 
activities such as scoping, testing, version control, evolution, management and
agile practices \citep{vale2014splice}. \ac{SPLICE} is part of the \acf{RiSE}
\citep{Almeida2004}, formerly called \ac{RiSE} Project, whose goal is to develop a robust framework for software reuse in order to enable the adoption of a 
reuse program. 

The tool \ac{SPLICE} already supports the specification of features and use cases. In order to accomplish 
the goal of this dissertation, we propose the extension of \ac{SPLICE} so that it will support the \ac{SPL} 
requirements specification activities stablished in the \ac{FeDRE} approach. The new version of the tool 
must enable the requirements engineers involved in this phase, to specify the \ac{SPL} requirements following 
the gidelines proposed in the \ac{FeDRE} approach, while providing guidance, and a reduction of effort and 
mistakes as the \ac{SPL} scope scales.
 
\section{Out of Scope}
\label{sc:outofscope}
The following topics are not considered in the scope of this dissertation: 
\begin{itemize}
\item \textbf{SPL Domain Requirements Evolution}

Although an approach has already been proposed for the \ac{SPL} domain requirements
evolution phase \ac{FeDRE2}, we still do not support this approach, but it is certainly a 
direction we intend to follow in the future.
\item \textbf{SPL Application Requirements Engineering}

In this work we do not consider the \ac{SPL} Application Engineering process, then our contributions do 
not cover  the \ac{SPL} Application Requirements Engineering.
\item \textbf{Non-SPL Tools}

This work is concerned with Software Product Lines development and tools and
environments that support the \ac{SPL} approach. Non-SPL tools are out of scope.
\end{itemize}

\section{Statement of the Contributions}
\label{sc:contributions}
As a result of the work presented in this dissertation, the following contribution can be highlighted:
\begin{itemize}
\item \textbf{Tool support for a SPL domain requirements specification approach
(FeDRE)} 
We extended the tool \ac{SPLICE}, a \ac{SPL} lifecycle management tool
and automated \acf{FeDRE}, thus improving the automation of Software Product Lines (\ac{SPL}) requirements engineering phase.
\end{itemize}

\section{Research Design}
\label{sc:design}

The first step of our work was to investigate the software product line area. This informal 
study also included to understand the requirements engineering phase for single systems and 
software product lines. As a result, we could write out the second chapter with some foundations 
on these subjects.
 
During the informal study we identified the need for tools that appropriately support the domain 
requirements engineering phase of software product lines. After choosing a requirements specification 
approach (\ac{FeDRE}), we extended an existing \ac{SPL} lifecycle management tool (\ac{SPLICE}) providing tool support 
for this approach.

In order to evaluate the proposed tool, we conducted a survey to identify limitations and needed 
improvements for the tool.  

\section{Dissertation Structure}
\label{sc:structure}
The remainder of this dissertation is organized as follows:

\begin{itemize}
\item \textbf{ Chapter \ref{ch:background} } reviews the essential topics
related to this work: Software Product Lines \ac{SPL}; requirements
engineering; \ac{SPL} requirements engineering; and \ac{SPLE} tool support.

\item \textbf{ Chapter \ref{ch:tool} } describes the tool \ac{SPLICE}, its
architeture and the set of frameworks and technologies used during its development. Also,  presents the new functional and non-functional 
requirements proposed for \ac{FeDRE} implementation based upon \ac{SPLICE}.

\item \textbf{ Chapter \ref{ch:survey} } describes an evaluation of \ac{FeDRE}
implementation.

\item \textbf{ Chapter \ref{ch:conclusion} } provides the concluding remarks. It
discusses our contributions, limitations, threats to validity, and outlines directions for future work.




\end{itemize}

