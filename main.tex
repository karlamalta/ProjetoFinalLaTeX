%% RiSE Latex Template - version 0.5
%%
%% RiSE's latex template for thesis and dissertations
%% http://risetemplate.sourceforge.net
%%
%% (c) 2012 Yguaratã Cerqueira Cavalcanti (yguarata@gmail.com)
%%          Vinicius Cardoso Garcia (vinicius.garcia@gmail.com)
%%
%% This document was initially based on UFPEThesis template, from Paulo Gustavo
%% S. Fonseca.
%%
%% ACKNOWLEDGEMENTS
%%
%% We would like to thanks the RiSE's researchers community, the 
%% students from Federal University of Pernambuco, and other users that have
%% been contributing to this projects with comments and patches.
%%
%% GENERAL INSTRUCTIONS
%%
%% We strongly recommend you to compile your documents using pdflatex command.
%% It is also recommend use the texlipse plugin for Eclipse to edit your documents.
%%
%% Options for \documentclass command:
%%         * Idiom
%%           pt   - Portguese (default)
%%           en   - English
%%
%%         * Text type
%%           bsc  - B.Sc. Thesis
%%           msc  - M.Sc. Thesis (default)
%%           qual - PHD qualification (not tested yet)
%%           prop - PHD proposal (not tested yet)
%%           phd  - PHD thesis
%%
%%         * Media
%%           scr  - to eletronic version (PDF) / see the users guide
%%
%%         * Pagination
%%           oneside - unique face press
%%           twoside - two faces press
%%
%%		   * Line spacing
%%           singlespacing  - the same as using \linespread{1}
%%           onehalfspacing - the same as using \linespread{1.3}
%%           doublespacing  - the same as using \linespread{1.6}
%%
%% Reference commands. Use the following commands to make references in your
%% text:
%%          \figref  -- for Figure reference
%%          \tabref  -- for Table reference
%%          \eqnref  -- for equation reference
%%          \chapref -- for chapter reference
%%          \secref  -- for section reference
%%          \appref  -- for appendix reference
%%          \axiref  -- for axiom reference
%%          \conjref -- for conjecture reference
%%          \defref  -- for definition reference
%%          \lemref  -- for lemma reference
%%          \theoref -- for theorem reference
%%          \corref  -- for corollary reference
%%          \propref -- for proprosition reference
%%          \pgref   -- for page reference
%%
%%          Example: See \chapref{chap:introduction}. It will produce 
%%                   'See Chapter 1', in case of English language.

\documentclass[en,twoside,onehalfspacing,bsc]{risethesis}

\usepackage{natbib}
\usepackage{babel}
\usepackage{supertabular}
\usepackage{microtype}
\usepackage{lscape}
 


%% Change the following pdf author attribute name to your name.
\usepackage[linkcolor=blue,citecolor=blue,urlcolor=blue,colorlinks,pdfpagelabels,pdftitle={Karla
Malta's Bachelor Thesis},pdfauthor={Karla Malta}]{hyperref}

\address{SALVADOR}

\universitypt{Universidade Federal da Bahia}
\universityen{Federal University of Bahia}

\departmentpt{Depertamento de Ci\^{e}ncia da Computa\c{c}\~{a}o}
\departmenten{Computer Science Department}

\programpt{Programa Multiinstitucional de P\'{o}s-gradua\c{c}\~{a}o em Ci\^{e}ncia da Computa\c{c}\~{a}o}
\programen{Bachelor's Degree in Computer Science}

\majorfieldpt{Ci\^{e}ncia da Computa\c{c}\~{a}o}
\majorfielden{Computer Science}

\title{SPLICE-FeDRE: a SPL Domain Requirements Specification Tool}
\date{November/2015}

\author{Karla Malta Amorim da Silva}
\adviser{Eduardo Santana de Almeida}
\coadviser{Raphael Pereira de Oliveira}

\begin{document}

\frontmatter
\frontpage
\presentationpage

\begin{dedicatory}
I dedicate this dissertation to my family, friends and
professors who gave me all necessary support to get here.
\end{dedicatory}

%%\acknowledgements
%%Agradeço a Deus por ter me dado oportunidade de realizar este trabalho,
% saúde
%%e energia para superar todas as dificuldades. Também por cada uma das pessoas
%%citadas abaixo, que de alguma forma contribuíram para que essa conquista
% fosse possível.

%%Agradeço aos meus pais, José Carlos e Suely, por tudo que me ensinaram, por
%%acreditarem em mim e por todo suporte e incentivo para a minha formação
%%pessoal e profissional. À minha irmã, Karoline, pela sinceridade e amizade.
% Ao meu namorado, Heitor, pela cumplicidade, carinho e companhia.

%%Agradeço ao meu orientador, Dr. Eduardo Almeida, por investir em mim e guiar
% o meu trabalho. Agradeço também ao meu co-orientador, Dr. Raphael Oliveira, pela
%%disponibilidade e por compartilhar seus conhecimentos e experiências de vida.

%%Agradeço a todos os docentes do Departamento de Ciência da Computação com
%%quem tive a oportunidade de aprender durante os anos do curso de Ciência da
%%Computação, especialmente à Dra. Aline Andrade, pelos seus conselhos e
%%orientação. Agradeço ao Dr. Luciano Oliveira, pela excelência do seu
%%trabalho como professor e  coordenador do curso.  Agradeço também ao Dr.
% John McGregor por me receber no seu grupo de pesquisa e orientar durante o meu
%%estágio de verão na Universidade de Clemson (USA), onde fiz minha
% graduação sanduíche.

%%Agradeço aos meu amigos que influenciaram positivamente e participaram direta
% ou indiretamente da minha formação: Nanci Bonfim, Melissa Wen, Laiza
% Camurugy,
%%Gabriel Uaquim, Rodrigo Vieira, Félix Farias, Caio Almeida e Diego Arize.
%%Agradeço também à minha amiga Marie Cudd por me adotar em Clemson e marcar
%%minha vida de uma forma extraordinária.

%%Agradeço à Universidade Federal da Bahia por me proporcionar esta
% formação, à Fapesb, pela bolsa de Iniciação a Extensão, à Capes pela bolsa do
%%programa Ciência sem Fronteiras, e às instituições onde estagiei,
% EcGlobalPanel, IFBA, TecnoTrends e Ericsson Inovação. 


\begin{epigraph}[]{Proverbs 3:13}
Blessed is the man who finds wisdom, the man who gains understanding. 
\end{epigraph}

\resumo
% Escreva seu resumo no arquivo resumo.tex
Linha de Produto de Software (LPS) é uma metodologia para o desenvolvimento de uma diversidade 
de produtos de software relacionados e sistemas com uso intensivo de software. Durante o desenvolvimento de 
uma LPS, uma ampla variedade de artefatos é criada para ser reusável ao longo 
do desenvolvimento de cada sistema da linha de produto.

Requisitos são um exemplo destes artefatos reusáveis que podem ser instanciados e adaptados para derivar os 
requisitos de produtos específicos. Gerir requisitos em LPS é uma tarefa árdua porque eles são complexos, 
interligados, e divididos em comuns, variáveis e requisitos de um produto específico.  Assim, o processo de 
engenharia de requisitos deve ter suporte ferramental para controlar
a complexidade e o grande volume de requisitos elicitados.

Neste trabalho, propomos uma ferramenta de suporte para realizar a especificação dos requisitos em LPS de forma 
sistemática, através do uso de diretrizes, mostrando passo a passo como a especificação deve ser feita.

\begin{keywords}
linha de produto de software, especificação de requisitos, ferramenta
\end{keywords}

\abstract
% Write your abstract in a file called abstract.tex
Software Product Line (SPL) is a methodology for developing a diversity of related 
software products and software-intensive systems. During the development of a SPL, a 
wide range of artifacts are created to be reusable throughout the development of 
each system within the product line.

Requirements are an example of these reusable artifacts that can be instantiated and adapted to 
derive the requirements for individual products. Managing SPL requirements is a hard task because 
the are complex, interlinked, and divided into common, variable and product-specific requirements. 
Thus, the requirements engineering process must be tool-supported to handle complexity and the huge 
volume of elicited requirements.

In this work, we propose a support tool for performing the specification of the SPL requirements in 
a systematic way through the use of guidelines,  showing step by step how the specification should be done.

\begin{keywords}
software product line, requirements specification, tool
\end{keywords}

% Summary (tables of contents)
\tableofcontents

% List of figures
\listoffigures

% List of tables
\listoftables

% List of acronyms
% Acronyms manual: http://linorg.usp.br/CTAN/macros/latex/contrib/acronym/acronym.pdf
\listofacronyms
\input{acronyms}

% List of listings
%\lstlistoflistings

\mainmatter

\chapter{Introduction}
\label{ch:introduction} 
A \acf{SPL} is outlined as a collection of similar software intensive systems
that share a set of common features satisfying the wants of specific customers, market segments 
or mission. Those similar software systems are developed from a set of core assets, comprised of 
documents, specifications, components, and other software artifacts that may be reusable throughout 
the development of each system within the product line
\citep{rafael2013systems}.

Requirements are typical assets in \ac{SPL}. They are specified in reusable models,
in which commonalities and variabilities are documented explicitly. Thus, these 
requirements can be instantiated and adapted to derive the requirements for an 
individual product \citep{cheng2007research}. New products in the SPL will be
much simpler to specify, because the requirements are reused and tailored
\citep{clements2002software}.

\acf{RE} in \ac{SPL} has an additional cost. Many \ac{SPL} requirements are
complex, interlinked, and divided into common, variable and product-specific requirements 
\citep{birk2003report,de2014defining}.  The requirements engineering process
must be tool-supported to handle complexity and the huge volume of elicited requirements
\citep{birk2003report}.

The focus of this dissertation is to provide a support tool for performing the specification of the 
\ac{SPL} requirements in a systematic way through the use of guidelines,  showing step by step how the 
specification should be done.

This chapter contextualizes the focus of this dissertation and starts by
presenting its motivation in \secref{sc:motivation} and a clear definition of the problem in 
\secref{sc:problem}. A brief overview of the proposed solution is presented in
\secref{sc:related}, while \secref{sc:outofscope} describes some aspects that
are not directly addressed by this work.
\secref{sc:contributions} presents the main contributions,  
\secref{sc:design} presents the research design  and, finally,
\secref{sc:structure} outlines the structure of this dissertation.

\section{Motivation}
\label{sc:motivation}
Within the \ac{SPL} paradigm, it is very important to perform a good requirements
engineering phase, because it is the basis of the  \ac{SPL} paradigm. However, existing 
tools are not designed to support the requirements engineering process for software 
product lines. Existing tools support only single product development and therefore 
lack support for modeling commonalities and variabilities as well as variation points in 
requirements \citep{birk2003report}.

Some approaches have been proposed to perform the specification and evolution of
the \ac{SPL} requirements in a systematic way through the use of guidelines: 
\acf{FeDRE} and \acf{FeDRE2}.
These approaches are considered easy to use and useful, however, they do not have a support tool. 
The lack of tool support can lead to mistakes during the manual execution of the guidelines, moreover, 
without a tool support these approaches can have problems with scalability.

In this sense, a \ac{SPL} Requirements Engineering tool is proposed to automatize the
\ac{SPL} requirements specification activities according to the \ac{FeDRE} approach. This tool is 
an extension of the tool \ac{SPLICE} \citep{splice2014cbsof}, which is an
integrated tool for developing \ac{SPL}.

\section{Problem Statement}
\label{sc:problem}

This work investigates the problems of complexity and scalability in \ac{SPL}
requirements specification phase to understand its activities in order to improve 
automation of these activities. This work promotes effort and mistakes reduction during 
\ac{SPL} requirements specification by poviding a \ac{SPL} Requirements Engineering tool .   

\section{Related Work}
\label{sc:related}
\acf{FeDRE} \citep{de2014defining} was 
defined and evaluated to aid developers in the \acf{RE} activity for \ac{SPL} 
development. The \ac{FeDRE} focus is the requirements specification in the Domain Engineering activity. 
\ac{FeDRE} realizes chunks of features from a feature model into functional requirements, which are then 
specified by use cases. Also, it provides detailed guidelines on how to specify the requirements.  
A first evaluation of \ac{FeDRE} was performed through an empirical study within a \ac{SPL} project, where \ac{FeDRE} 
was perceived as easy to learn and useful by the participants.

\acf{SPLICE} is a web-based \ac{SPL} life-cycle 
management tool that provides traceability and variability management and supports most of the \ac{SPL} process 
activities such as scoping, testing, version control, evolution, management and
agile practices \citep{vale2014splice}. \ac{SPLICE} is part of the \acf{RiSE}
\citep{Almeida2004}, formerly called \ac{RiSE} Project, whose goal is to develop a robust framework for software reuse in order to enable the adoption of a 
reuse program. 

The tool \ac{SPLICE} already supports the specification of features and use cases. In order to accomplish 
the goal of this dissertation, we propose the extension of \ac{SPLICE} so that it will support the \ac{SPL} 
requirements specification activities stablished in the \ac{FeDRE} approach. The new version of the tool 
must enable the requirements engineers involved in this phase, to specify the \ac{SPL} requirements following 
the gidelines proposed in the \ac{FeDRE} approach, while providing guidance, and a reduction of effort and 
mistakes as the \ac{SPL} scope scales.
 
\section{Out of Scope}
\label{sc:outofscope}
The following topics are not considered in the scope of this dissertation: 
\begin{itemize}
\item \textbf{SPL Domain Requirements Evolution}

Although an approach has already been proposed for the \ac{SPL} domain requirements
evolution phase \ac{FeDRE2}, we still do not support this approach, but it is certainly a 
direction we intend to follow in the future.
\item \textbf{SPL Application Requirements Engineering}

In this work we do not consider the \ac{SPL} Application Engineering process, then our contributions do 
not cover  the \ac{SPL} Application Requirements Engineering.
\item \textbf{Non-SPL Tools}

This work is concerned with Software Product Lines development and tools and
environments that support the \ac{SPL} approach. Non-SPL tools are out of scope.
\end{itemize}

\section{Statement of the Contributions}
\label{sc:contributions}
As a result of the work presented in this dissertation, the following contribution can be highlighted:
\begin{itemize}
\item \textbf{Tool support for a SPL domain requirements specification approach
(FeDRE)} 
We extended the tool \ac{SPLICE}, a \ac{SPL} lifecycle management tool
and automated \acf{FeDRE}, thus improving the automation of Software Product Lines (\ac{SPL}) requirements engineering phase.
\end{itemize}

\section{Research Design}
\label{sc:design}

The first step of our work was to investigate the software product line area. This informal 
study also included to understand the requirements engineering phase for single systems and 
software product lines. As a result, we could write out the second chapter with some foundations 
on these subjects.
 
During the informal study we identified the need for tools that appropriately support the domain 
requirements engineering phase of software product lines. After choosing a requirements specification 
approach (\ac{FeDRE}), we extended an existing \ac{SPL} lifecycle management tool (\ac{SPLICE}) providing tool support 
for this approach.

In order to evaluate the proposed tool, we conducted a survey to identify limitations and needed 
improvements for the tool.  

\section{Dissertation Structure}
\label{sc:structure}
The remainder of this dissertation is organized as follows:

\begin{itemize}
\item \textbf{ Chapter \ref{ch:background} } reviews the essential topics
related to this work: Software Product Lines \ac{SPL}; requirements
engineering; \ac{SPL} requirements engineering; and \ac{SPLE} tool support.

\item \textbf{ Chapter \ref{ch:tool} } describes the tool \ac{SPLICE}, its
architeture and the set of frameworks and technologies used during its development. Also,  presents the new functional and non-functional 
requirements proposed for \ac{FeDRE} implementation based upon \ac{SPLICE}.

\item \textbf{ Chapter \ref{ch:survey} } describes an evaluation of \ac{FeDRE}
implementation.

\item \textbf{ Chapter \ref{ch:conclusion} } provides the concluding remarks. It
discusses our contributions, limitations, threats to validity, and outlines directions for future work.




\end{itemize}


\chapter{An Overview on Software Product Lines, Requirements Engineering, SPL
Requirements Engineering and SPLE Tool Support}
\label{ch:background}


This chapter presents fundamental information for the understanding of four
topics that are relevant to this work: software product lines, requirements
engineering, and \ac{SPL} requirements engineering. \secref{sc:productlines} discusses the
motivation, benefits, and the SPL development process.
\secref{sc:requirementsengineering} presents requirements engineering.
\secref{sc:splrequirementsengineering} presents \ac{SPL} requirements engineering.
\secref{sc:spltools} presents \ac{SPLE} Tool Support. Finally,
\secref{sc:summary} presents a summary of this chapter.


\section{Software Product Lines}
\label{sc:productlines}

\subsection{Introduction}
Nowadays we experience the age of customization, but it was not always like
that. There was a time when goods were handcrafted for individual costumers. 
Over the years, the number of people who could afford to buy several kinds of 
products has increased \citep{Pohl2005}. In order to meet this rising demand, 
the production line was invented, which enabled production for a mass market much 
more cheaply than individual product.

Customers were satisfied with mass produced products for a while \citep{Pohl2005}, 
however that kind of product lacks sufficient diversification to meet individual
customers’ wishes. Individualized products also have a drawback; they are a lot more expensive 
than standardized products. In that context, the industry was challenged to provide customized 
products at reasonable costs to satisfy the wishes of specific customers and market segments. 
The combination of mass customization and common platforms was the key to achieve that goal.

Mass customization is the large-scale production of goods tailored to individual
customers’ needs. It requires a higher technological investment which leads to higher prices for 
the individualized products and/or to lower profit margins for the company. The platform approach 
though, enables manufacturers to offer a larger variety of products and to reduce costs at the same 
time. A platform is defined as a base of technologies on which other technologies or processes are 
built. The combination of mass customization and a common platform allows us to reuse a common 
base of technology and to bring out products in close accordance with customers’ wishes \citep{Pohl2005}.

In the software domain, that combination resulted in a software development
paradigm called \acf{SPLE}. A Software
Product Line (\ac{SPL}) is a set of software-intensive systems that share a common, managed feature set, satisfying a particular market 
segment’s specific needs or mission and that are developed from a common set of core assets in a 
prescribed way \citep{clements2002software}.

\subsection{The Benefits}
Developing software under the Product Line Engineering paradigm offers many benefits for a company, 
some examples follow:
\begin{itemize}
\item \textbf{Reduction of Development Costs}

A good reason for applying the Product Line Engineering paradigm is the reduction of costs as the reuse 
of assets increases. Through the reuse of artifacts from the platform in different systems, the development 
of each of these systems becomes cheaper. First, the company has to invest in the development of the platform. 
Also, the way in which the artefacts from the platform will be reused has to be well planned beforehand. Then, 
from a certain point, called break-even point, the initial investment will be paid off. The precise location of 
this point is influenced by many characteristics of the company, the market it has envisaged, its customers, 
expertise, kinds of products, the way the product line is created and others. 

\begin{figure}[htp]
\begin{center}
  \includegraphics[width=11cm]{chapters/background/img/splcosts.png}
  \caption[Costs for developing systems as single systems compared to product 
  line engineering]{Costs for developing systems as single systems compared to product line engineering \citep{Pohl2005}}
  \label{fg:spl-costs}
\end{center}
\end{figure}

\figref{fg:spl-costs} shows that the costs to develop a few systems in an \ac{SPL} approach are higher than in a 
single systems approach. However, using product line engineering, the costs are significantly lower for larger 
systems quantities. 

\item \textbf{Quality improvement}

Creating products under the \ac{SPL} paradigm improves the quality of all products of a product family. 
The shared components from the platform are reviewed and tested in many products. They have to work properly in 
more than one kind of product. The extensive quality assurance indicates a significantly higher opportunity of 
detecting faults and correcting them, thereby improving the quality of all products \citep{Pohl2005}.

\begin{figure}[htp]
\begin{center}
  \includegraphics[width=11cm]{chapters/background/img/spl-timetomarket.png}
  \caption[Comparison of time to market with and without product line
  engineering]{Comparison of time to market with and without product line
  engineering \citep{Pohl2005}}
  \label{fg:spl-timetomarket}
\end{center}
\end{figure}

\item \textbf{Reduction of Time-to-market}

Another very important success factor for a product is the time to market.
\ac{SPL} engineering demands a high upfront investment, which makes time to
market initially higher if compared with to single-systems engineering. However, as the reuse of artefacts grow, 
the time to market is significantly shortened for new products, as can be seen
in \figref{fg:spl-timetomarket}.
\item \textbf{Reduction of Maintenace Effort}

When a reusable asset from the platform is changed, this change may be
propagated to all products in which it is being used. It usually leads to a simpler and cheaper maintenance and 
evolution, if compared to maintain and evolve a bunch of single products in a separate way.

\item \textbf{Benefits for the Customers}

The benefits for the customers are higher quality products at reasonable prices
because the production costs become lower in \ac{SPL} engineering. Besides, products are adapted to their 
real needs and wishes.
\end{itemize}


\subsection{The SPL Development Process}
There are a number of different definitions for the \acf{SPL} Development
Process on the literature. \citep{Pohl2005} introduced a framework for SPLE paradigm, shown in
\figref{fg:spl-pohlframework}. This framework is divided in two processes: 

\begin{figure}[htp]
\begin{center}
  \includegraphics[width=13cm]{chapters/background/img/pohl-framework.png}
  \caption[The software product line engineering framework]{The software product line engineering framework \citep{Pohl2005}}
  \label{fg:spl-pohlframework}
\end{center}
\end{figure}

\begin{itemize}
\item \textbf{Domain engineering:} This is the process that aims to establish a reusable 
platform and define the commonality and the variability of the product line. Domain Engineering 
is composed of five sub-processes: domain requirements, domain design, domain realization, domain 
testing, and product management \citep{Pohl2005}. 
\item \textbf{Application engineering:} This process is responsible for deriving
product line applications from the platform created in domain engineering, where the previously 
developed components are assembled to compose a product. The application engineering is composed 
of four sub-processes: application requirements engineering, application design, application realization, 
and application test \citep{Pohl2005}.  
\end{itemize}

Another popular  definition of the \acf{SPL} Development
Process  can be related to the aforementioned approach. \citep{clements2002software} 
defined three essential activities to Software Product Lines:
\textbf{\acf{CAD}}, \textbf{\acf{PD}} and \textbf{Management activity},
ilustrated in \figref{fg:spl-activities}.
In essence, \acf{CAD} activity is the Domain engineering process, 
and the \acf{PD} activity is the Application engineering process. 
The main difference between these approaches is the Management activity, which is not considered 
as a process in the first mentioned approach \citep{Pohl2005}. 

\begin{figure}[htp]
\begin{center}
  \includegraphics[width=10cm]{chapters/background/img/SPLactivities.png}
  \caption[SPL Activities]{SPL Activities \citep{clements2002software}}
  \label{fg:spl-activities}
\end{center}
\end{figure}

\subsubsection{Core Asset Development (Domain Engineering)}
\acf{CAD}, also called by \citep{Pohl2005} as domain engineering, is an activity 
that aims to develop assets to be further reused in other activities. In \figref{fg:spl-coreasset}, it is shown the core 
asset development activity, which is  interactive, and its inputs and outputs influence each other. The 
inputs of this activity are product constraints; production constraints; architectural styles; design 
patterns; application frameworks; production strategy and preexisting assets. This phase is composed of the 
following sub processes \citep{Pohl2005}:

\begin{itemize}
\item \textbf{Product Management} deals with the economic aspects associated with the software product line and in particular with the market strategy.
\item \textbf{Domain Requirements Engineering} involves all activities for eliciting and documenting the common and variable requirements of the product line.
\item \textbf{Domain Design} encompasses all activities for defining the reference architecture of the product line, 
\item \textbf{Domain Realization} deals with the detailed design and the implementation of reusable software components.
\item \textbf{Domain Testing} is responsible for the validation and verification of reusable components. 
\end{itemize}

\begin{figure}[htp]
\begin{center}
  \includegraphics[width=10cm]{chapters/background/img/SPLcoreasserts.png}
  \caption[Core Asset Development]{Core Asset Development \citep{clements2002software}}
  \label{fg:spl-coreasset}
\end{center}
\end{figure}

This activity have three outputs: \textbf{Product Line Scope}, \textbf{Core
Assets} and \textbf{Production Plan}. The Product Line Scope describes the
products that will compose the product line or that the product line can include. This description is recommended to be detailed and well 
specified, for example, including market analysis activities in order to determine the product 
portfolio and to encompass which assets and products will be part of the product line. This 
specification must be driven by economic and business reasons to keep the product line 
competitive \citep{rafael2013systems}. 

Core assets are the basis for production of products in the product line. It
includes an architecture that will fulfill the needs of the product line, specify 
the structure of the products and the set of variation points required to support the 
spectrum of products. It may also include components and their documentation \citep{clements2002software}. 

Lastly, the production plan describes how products are produced from the core
assets. It details the overall scheme of how the individual attached processes can be 
fitted together to build a product \citep{clements2002software}. It is what links all the 
core assets together, guiding the product development within the constraints of the product line.

\subsubsection{Product Development (Application Engineering)}

\begin{figure}[htp]
\begin{center}
  \includegraphics[width=10cm]{chapters/background/img/SPLproduct-development.png}
  \caption[Product Development]{Product Development \citep{clements2002software}}
  \label{fg:spl-productdev}
\end{center}
\end{figure}

The inputs for this activity are the outputs of the core asset development activity (product line scope, 
core assets, and production plan) and the requirements specification for individual products as seen in 
\figref{fg:spl-productdev}. The production plan guides how individual products within a product line are constructed using 
the core assets.

The outputs from this activity should be analyzed by the software engineer and the corrections must be fed 
back to the \acf{CAD} activity. During the product development process, some insights happen 
and it is important to report problems and faults encountered to keep the core asset base healthy.

\subsubsection{Management}

The management activity is responsible for the production strategy and is vital
for success of the product line \citep{Pohl2005}. It is performed in two levels: technical and 
organizational. The technical management supervise the CAD and PD activities by certifying that both 
groups that build core assets and products are focused on the activities they are supposed to, and follow 
the process. The organizational management must ensure that the organizational units receive the right 
resources in sufficient amounts \citep{clements2002software}.

\section{Requirements Engineering}
\label{sc:requirementsengineering}

Software requirements are descriptions of what the system is expected to do, the
services that it must provide and the constraints it must satisfy
\citep{Sommerville2011}.
Software requirements are usually classified in a classic way as functional and non-functional. 
Functional requirements describe what the system must do and non-functional requirements place constraints 
on how these functional requirements are implemented
\citep{sommerville2005integrated}.

According to \citep{sommerville1998requirements},
\acf{RE} is the process by which the  software requirements are defined. They
state that a process is an organized set of activities that transforms inputs to 
outputs. Thus, a complete description of a \ac{RE} process should include what
activities are carried out, the structuring or schedule of these activities, who is responsible for each 
activity and the tools used to support the \ac{RE} activities.   

The \ac{RE} lifecycle includes requirements elicitation, analysis, negotiation, specification, verification, and 
management, where \citep{clements2002software,sommerville2005integrated}:

\begin{itemize}
\item \textbf{Elicitation} identifies sources of requirements information and discovers the 
users’ needs and constraints for the system.
\item \textbf{Analysis} understands the requirements, their overlaps, and their conflicts.
\item \textbf{Negotiation} reaches agreement to satisfy all stakeholders,
solving conflicts that are identified.
\item \textbf{Specification} documents the user’s needs and constraints clearly and precisely.
\item \textbf{Verification} checks if the requirements are complete, correct, consistent, and clear.
\item \textbf{Management} controls the requirements changes that will inevitably arise.
\end{itemize}

\section{SPL Requirements Engineering}
\label{sc:splrequirementsengineering}

Requirements are typical assets in \ac{SPL}. They are specified in reusable
models, in which commonalities and variabilities are documented explicitly. Thus, these requirements 
can be instantiated and adapted to derive the requirements for an individual
product \citep{cheng2007research}.
During product derivation, for each variant asset, it is decided whether the asset is (or is not) supported by 
the product to be built. When a domain requirement is instantiated, it can become a concrete product requirement. 
Thus, new products in the \ac{SPL} will be much simpler to specify, because the
requirements are reused and tailored \citep{clements2002software}. 

Deciding which products to build depends on business goals, market trends,
technological feasibility, and so on. On the other hand, there are many sources of information 
to be considered and many trade-offs to be made. The \ac{SPL} requirements must be general enough to support 
reasoning about the scope of the \ac{SPL}, predicting future changes in
requirements and anticipated \ac{SPL} growth.

In practice, establishing the requirements for an \ac{SPL} is an iterative and
incremental effort, covering multiple requirements sources with many feedback loops and validation activities 
\citep{chastek2001product}. Thus, \acf{RE} in \ac{SPL} has an
additional cost. Many \ac{SPL} requirements are complex, interlinked, and divided
into common, variable and product-specific requirements \citep{birk2003report, 
de2014defining}. Regarding to single systems, \ac{RE} for \ac{SPL} has
some differences, such as \citep{clements2002software,
Pohl2005, thurimella2007evolution}:

\begin{itemize}
\item \textbf{Elicitation} captures anticipated variations over the foreseeable life-cycle of the 
\ac{SPL}. \ac{RE} must anticipates prospective changes in requirements, such as
laws, standards, technology changes, and market needs for future products. Thus, its sources of information are probably larger 
than for single-system requirements elicitation.
\item \textbf{Analysis} identifies variations and commonalities, and discovers opportunity for reuse.
\item \textbf{Negotiation} solves conflicts not only from a logical viewpoint, but also taking into 
consideration economical and market issues. The \ac{SPL} requirements may
require sophisticated analysis and intense negotiation to agree on both common requirements and variation 
points that are acceptable for all the systems.
\item \textbf{Specification} documents a \ac{SPL} set of requirements. Notations are used to represent the 
product line variabilities and enable the product instantiation.
\item \textbf{Verification} checks if the \ac{SPL} requirements can be instantiated for the products, 
ensuring the reusability of the requirements.
\item \textbf{Management} must provide a systematic mechanism for proposing changes, evaluating how the 
proposed changes will impact the \ac{SPL}, specifically its core asset base. Evolution can affect the reuse and 
customization, therefore, appropriate mechanisms must be used o manage the variabilities.
\end{itemize}

In \ac{SPL}, \ac{RE} also has influence of several stakeholders that participate
of the \ac{SPL}. Identifying stakeholders that directly influence the \ac{RE} is
essential to define the requirements negotiation participants. They are responsible for 
resolving conflicts and providing information. 

Each stakeholder plays a role with respect to the \ac{SPL}. Many of the stakeholders
that help to define the requirements also use them. These users have different expectations of 
the outputs of \ac{SPL} analysis. Some may simply want to confirm that their interests have been represented 
(e.g., marketers, domain expert and analyst domain). Others ( e.g., architects and developers) may want to 
describe proposed functional and non-functional capabilities, and their commonality and variability across 
the \ac{SPL}, thus, those decisions about architectural solutions and asset construction should be taken into account 
\citep{chastek2001product}.

Several approaches to deal with the definition and specification of functional
requirements in \ac{SPL} development have been proposed over the last few years. Some approaches specify 
the \ac{SPL} requirements through features and use cases
\citep{griss1998integrating, bayer2000customizable, moon2005approach,
eriksson2005pluss, bonifacio2009modeling, alferez2011supporting,
mussbacher2012aourn, shaker2012feature, de2014defining}.
A \ac{SPL} functional requirement represented as an use case has at least the following fields: identifier, name, 
description, associated feature(s), pre and post-conditions, and the main
success scenario , as shown in Table \ref{table:use-case}. It may also have
alternative scenarios, includes/extends relationships, and so on. The feature associated to the use case handles the 
variability within the \ac{SPL}.

\begin{table}[]
\centering
\scriptsize
\caption{SPL Use Case Example (Addapted from \citep{de2014defining})}
\label{table:use-case}
\begin{tabular}{|l|l|l|l|}
\hline
\multicolumn{1}{|c|}{\textbf{*ID:}} & \multicolumn{3}{l|}{Use case identifier} \\ \hline 

\multicolumn{1}{|c|}{\textbf{*Name:}} & \multicolumn{3}{l|}{Use case name} \\ \hline 

\multicolumn{1}{|c|}{\textbf{*Description:}} & \multicolumn{3}{l|}{Use case description} \\ \hline 

\multicolumn{1}{|c|}{\textbf{*Associated feature:}} & Feature associated to the
use case & \textbf{Actor(s) [0..*]:} & Actor associated to the use case \\ \hline

\multicolumn{1}{|c|}{\textbf{*Pre-condition:}} & Use case pre-condition &
\textbf{*Post-condition:} & Use case post-condition \\ \hline

\multicolumn{4}{|c|}{\textbf{*Main Success Scenario}} 
\\ \hline

\multicolumn{1}{|l|}{\textbf{Step}} & \multicolumn{1}{|l|}{\textbf{Actor
Action}} & \multicolumn{2}{|l|}{\textbf{Blackbox System Response}} \\ \hline 

\multicolumn{1}{|l|}{Step represented by a number} & \multicolumn{1}{|l|}{Actor
action} & \multicolumn{2}{|l|}{System response} \\ \hline
\end{tabular}
*Mandatory Fields
\end{table}

However, most of the approaches for specifying \ac{SPL} functional requirements
do not propose guidelines, showing step by step how the specification should be
done. This lack of guidelines may lead to some challenges and risks \citep{de2014defining}.

\subsection{Risks and Challenges}

A key \ac{RE} challenge for \ac{SPL} development includes strategic and
effective techniques for analyzing domains, identifying opportunities for \ac{SPL}, and identifying 
the commonalities and variabilities of an \ac{SPL} \citep{cheng2007research}. Another challenge 
related to \ac{RE} is that the applicability of more systematic techniques and tools is limited, 
partly because such techniques are not yet designed to cope with \ac{SPL} development’s inherent 
complexities \citep{birk2003report}.

Regarding to the risks associated with \ac{RE} for \ac{SPL}, the major risk is failure to
capture the right requirements, and their variabilities, over the life of the \ac{SPL} 
\citep{clements2002software}. Documenting the wrong or inappropriate
requirements, failing to keep the requirements up-to-date, or failing to document the requirements at all, 
may affects the subsequent activities (architecture, implementation, tests, and so  on). They will 
be unable to produce systems that satisfy the customers and fulfill the market expectations. 
Moreover, inappropriate requirements can result from the  following
\citep{clements2002software}:

\begin{itemize}
\item \textbf{Failure in the communication between core assets requirements
development and product requirements development.} The core asset builders need to 
know the requirements they must build, while the product-specific software builders 
must know what is expected of them. The lack of communication between these two development 
stages may lead to inconsistent requirements or even unnecessary variabilities
in the requirements.
\item \textbf{Insufficient generality.} Insufficient generality in the
requirements leads to a design that is too fragile to deal with the change actually experienced 
over the life-cycle of  the \ac{SPL}.
\item \textbf{Excessive generality.} Excessive generality on requirements leads
to excessive effort in producing both core assets (to provide that generality) and specific 
products (which must turn that generality into a specific instantiation).
\item \textbf{Wrong variation points.} Incorrect determination of the variation
points results in inflexible products and the inability to respond rapidly to customer needs and market shifts.
\item \textbf{Failure to account for qualities other than behavior.} \ac{SPL}
requirements (and software requirements in general) should capture requirements for quality 
attributes such as performance, reliability, and security.
\end{itemize}

\section{SPLE Tool Support}
\label{sc:spltools}

Since the early days of computer programming, software engineers use a variety of tools to support 
software development. \acf{SE} tools and environments are becoming progressively
important as the demand for software, its diversity and complexity increases. The computer industry is a competitive 
industry and there is a pressure to produce software at lower costs and faster because time-to-market is a 
decisive factor for success. Thus, modern software engineering cannot be accomplished without reasonable 
tool support \citep{ossher2000software}.

The commercial potential of the \ac{SPL} approach has already been demonstrated in
numerous case studies. While product line development is increasingly accepted, professional 
tool support is still insufficient and represents a key challenge for future research 
\citep{Pohl2005,schmid2006requirements}.

\acf{SPLE} tool support focuses almost exclusively on a single, cross-cutting
aspect of \ac{SPLE}: variability management \ac{VM}, or making software and
artifacts (such as requirements, tests, and documentation) configurable in a way that they can be  developed together, while each 
product still receives its specifically adapted version \citep{schmid2013product}.  Thus, an effective 
and efficient variability management \ac{VM} is the base of the successful reuse of development 
artifacts \citep{boutkova2011experience}.

\acf{VM} tools support four main activities: modeling variability, modeling the
relationship between variability and a generic artifact, supporting configuration of generic artifacts, 
and deriving customized products \citep{schmid2013product}.

The requirements engineering process must be tool-supported to handle the huge
volume of elicited requirements. There are several differences between a single 
product development and a product line development and therefore a tool must be 
capable to support that development, including the additional activities that must be 
performed in the requirements engineering phase. However, existing tools are not designed 
to support the requirements engineering process for software product lines. Existing 
tools support only single product development and therefore lack support for modeling 
commonalities and variabilities as well as variation points in requirements
\citep{birk2003report}.

\section{Summary}
\label{sc:summary}

In this chapter, we discussed about important concepts to this work: the area of
\acf{SPL}, \acf{RE} , \ac{SPL} Requirements Engineering and \ac{SPLE} tool
support.

Next chapter presents an extension of \acf{SPLICE}, a web-based, collaborative
support tool for the \ac{SPL} lifecycle steps.


\chapter{SPLICE-FeDRE: a SPL Domain Requirements Specification Tool}
\label{ch:splice}

\section{Introduction}

In this chapter, we present functional and non-functional requirements for a tool we call SPLICE-FeDRE, 
and its implementation. The tool is an extension of \acf{SPLICE}, built in order
to support and integrate \ac{SPL} activities, such as, requirements management,
architecture, coding, testing, tracking, and release management, providing process automation and 
traceability across the process.

The remainder of this chapter is organized as follows: \secref{sc:fedre}
presents the \ac{FeDRE} approach; \secref{sc:splice} describes the tool
\ac{SPLICE}; \secref{sc:requirements} presents the requirements of SPLICE-FeDRE;
details of the implementation of SPLICE-FeDRE are discussed in
\secref{sc:implementation}; \secref{sc:operation} shows the tool SPLICE-FeDRE
in operation; and, finally, \secref{sc:solutionsummary} presents the summary of
the chapter.
 
\section{FeDRE Overview}
\label{sc:fedre}

The \acf{FeDRE} approach \citep{de2014defining} for 
\ac{SPL} has been defined by considering the feature model as the main artifact for specifying \ac{SPL} requirements. 
The aim of the approach is to perform the requirements specification by systematically utilizing the features 
identified in the \ac{SPL} domain through the use of guidelines that establish traceability links between features 
and requirements.

The main activities of the \ac{FeDRE} approach are: Scoping and Requirements Specification for Domain Engineering. 
\figref{fg:fedre-activities} shows the activities of \ac{FeDRE}, which are
detailed in this chapter. The following roles are involved in these activities: Domain Analyst, Domain Expert, Market Expert and the Domain Requirements Analyst.

\begin{figure}[htp]
\begin{center}
  \includegraphics[width=11cm]{chapters/proposed_solution/img/captures/fedre_activities.png}
  \caption[Overview of the FeDRE approach]{Overview of the FeDRE approach \citep{de2014defining}}
  \label{fg:fedre-activities}
\end{center}
\end{figure}

\subsection{Scoping}

The first activity performed in \ac{FeDRE} is the Scoping. This determines not only
what products to include in an \ac{SPL} but also whether or not an organization should 
launch the \ac{SPL}. Three main artifacts are produced as a result of the Scoping activity: 
the Feature Model, the Feature Specification, and the Product Map, using the Existing 
Assets (if any) as the input artifact. These three artifacts will drive the \ac{SPL} requirements 
specification for domain engineering. Each of these artifacts (input and outputs) is 
detailed below.

\subsubsection{Existing Assets}

Existing assets (e.g., user manual or existing systems) help the Domain Analyst
and the Domain Expert to identify the features and products in the \ac{SPL}. When they do not 
exist, a proactive approach can be followed to build the \ac{SPL} from scratch.

\subsubsection{Feature Model}

Feature modeling is a technique that is used to model common and variable
properties, and can be used to capture, organize and visualize features in the \ac{SPL}.

\subsubsection{Feature Specification}

The Domain Analyst is responsible for specifying the features using a feature
specification template. This template captures the detailed information of the
features and maintains traceability with all the artifacts involved.

\subsubsection{Product Map}

Each of the identified features is assigned to the corresponding products in the \ac{SPL}. 
The set of relationships among features and products produces the Product Map artifact, 
which describes all the features that are required to build a specific product in the \ac{SPL}.
All these artifacts are the input for the Requirements Specification for Domain Engineering 
activity, which is described next.

\subsection{Requirements Specification for Domain Engineering}

This activity specifies the \ac{SPL} requirements for domain engineering. These requirements 
allow realization of the features and desired products identified in the Scoping activity. 
The steps required to perform this activity are described in the Guidelines for Specifying 
SPL Functional Requirements, Sub-section \ref{subsec:guidelines} below.

This activity, seen in \figref{fg:fedre-activities}, uses the Feature Model,
Feature Specification and Product Map as input artifacts and produces the Glossary, Functional Requirements and Traceability Matrix 
as output artifacts. Each of these output artifacts is detailed below.

\subsubsection{Glossary}

The Glossary describes and explains the main terms in the domain in order to
provide the stakeholders with a common vocabulary and avoid misconceptions.

\subsubsection{Functional Requirements}

This artifact contains all the functional requirements identified (common or
variable), for the family of products that constitute the \ac{SPL}. Use cases are used 
to specify the \ac{SPL} functional requirements. Each functional requirement has a unique 
Use case id, a Name, a Description, Associated Feature(s), Pre and Post-Conditions, and 
the Main Success Scenario. A functional requirement can also be related to an Actor and 
may have Include and/or Extend relationships with other use case(s).

\subsubsection{Traceability Matrix}

The Traceability Matrix is a matrix that contains the links among features and
the functional requirements.

\subsection{Guidelines for Specifying SPL Functional Requirements}
\label{subsec:guidelines}

The purpose of the guidelines is to guide the Requirements Analyst in 
the specification of \ac{SPL} functional requirements for domain engineering. The 
guidelines have been structured to specify functional requirements by addressing the following 
questions: i) Which features or set of features will be grouped to be specified by use 
cases? ii) What are the specific use cases for the feature or set of features? iii) Where 
should the use cases be specified? (when there is a set of features in a hierarchy, do we specify 
the use cases for each individual feature or only for the parent features?) and iv) How is the use 
case specified in terms of steps?

Activities, tasks and steps are used in the process of specifying requirements
for \ac{SPL}. \figref{fg:fedre-guide} shows the guidelines with the detailed
steps of each task for specifying \ac{SPL} functional requirements.

\begin{figure}[htp]
\begin{center}
  \includegraphics[width=11cm]{chapters/proposed_solution/img/captures/fedre_guide.png}
  \caption[Guidelines For Specifying SPL Functional Requirements]{Guidelines For Specifying SPL Functional Requirements \citep{de2014defining}}
  \label{fg:fedre-guide}
\end{center}
\end{figure}

\section{SPLICE Overview}
\label{sc:splice}

\acf{SPLICE} \citep{splice2014cbsof}  is an open source (GNU General Public
License), Python,  web-based software product line lifecycle management tool, providing 
traceability and variability management and supporting most of the \ac{SPL} process 
activities such as scoping, requirements, architecture, testing, version control, 
evolution, management and agile practices. This tool assists the engineers involved 
in the process, with the assets creation and maintenance, while providing traceability 
and variability management, as well offering detailed reports and enabling engineers 
to easily navigate between the assets using the traceability links.

\subsection{Metamodel}

\ac{SPLICE} proposes a lightweight metamodel, representing the interactions among \ac{SPL}
assets, developed in order to provide a way of managing traceability and variability. 
The proposed metamodel represents the reusable assets involved in a \ac{SPL} project, and 
simplified description of the models is presented next.

\begin{itemize}
\item \textbf{Scoping Module} comprises the Feature and the Product Model. Many artifacts relates 
directly with the Feature Model including Use Case, Glossary, User Story and Scope Backlog. 
A Product is composed of one or more Features.

\item \textbf{Requirements Module} involves the requirements engineering
traceability and interactions issues, considering the variability and commonality in the \ac{SPL} 
products. The main object of this \ac{SPL} phase is Use Case. The concept of User stories is used 
in this metamodel to represent what a user does or needs to do as part of his or her job function. 
 
\item \textbf{Testing Module} is composed of a name, description, the
Expected result and a set of Test Steps. One Test Case can have many Test Execution that 
represent one execution of it. The reasoning for the Test Execution is to enable a test 
automation machinery. The metamodel also represents the acceptance testing with the Acceptance 
Test and Acceptance Test Execution. 

\item \textbf{Agile Planning Module} contains Sprint Planning models, which are composed of a number of Tickets, a
deadline, an objective and a start date. At the end of the sprint, it happens a retrospective, 
represented in the model by Sprint Retrospective, that contains a set of Strong Points and Should be 
Improved models that express what points in the spring was adequate, and what needs improvement.
\end{itemize}

\subsection{Main Functionalities}

The main functionalities of \ac{SPLICE} include:

\begin{itemize}
\item  \textbf{Metamodel Implementation.} All the screens are completely
auto-generated based on the models descriptions, allowing the Software Engineer to 
easily modify the process. For every model, a complete \acf{CRUD} system is
created. The \ac{SPLICE} also provides advanced features such as filtering and classification.
\item  \textbf{Issue Tracking.} \ac{SPLICE} has a full-featured Issue Tracking. It
was extended to implement \ac{SPL} specific features and to provide traceability between other assets.
\item  \textbf{Traceability.} \ac{SPLICE} provides total traceability for all assets
in the metamodel, and is able to report direct and indirect relations between them. In reports, assets 
have hyperlinks, enabling the navigation between them.
\item  \textbf{Custom SPL Widgets.} \ac{SPLICE} has a set of custom widgets to
represent specific \ac{SPL} models. Such as Feature Map, Product Map, and Agile Poker planning.
\item  \textbf{Change history and Timeline.} \ac{SPLICE} has a rich set of features
to visualize how the project is going, where the changes are happening, and who did it. For every 
Issue or Asset, a complete Change history is recorded.
\item  \textbf{Unified Control Panel.} The tool aggregates the configuration of
all external tools in a unified interface. With the same credentials, the user is able to access all 
\ac{SPLICE} features, including external tools as \acf{VCS}.
\item  \textbf{Agile Planning.} The \ac{SPLICE} supports a set of Agile practices
such as effort estimation, where team members use effort and degree of difficulty to estimate 
their own work. The Features can be dragged by the mouse, and their position is updated in accordance.
\item  \textbf{Automatic reports generation.} \ac{SPLICE} has the ability of creating
reports, including PDFs. The generated report includes a cover, a summary and the set of 
the chosen artifact related to the product. This format is suitable for the requirements 
validation by stakeholders. The tool is also able to collect all reports for a given Product, and 
create a compressed file containing the set of generated reports.
\end{itemize}

\section{SPLICE-FeDRE Requirements}
\label{sc:requirements}
In the SPLICE-FeDRE specification, the following functional requirements were
defined:

\begin{itemize}
\item  \textbf{FR1 - FeDRE Feature Specification.} The tool should provide a complete 
\ac{CRUD} (Create, Read, Update and Delete) for the model Feature that satisfies the \ac{FeDRE} 
approach needs. The model Feature should include a unique Feature id, name, description, 
priority (high, medium or low), type (abstract or concrete), variability (mandatory, optional, OR ou XOR), 
binding time (compile, runtime),  parent feature, glossary, use case diagram, similar feature(s), 
required feature(s) and excluded feature(s).

\item  \textbf{FR2 - FeDRE Use Case Specification.} The tool should provide a
complete \ac{CRUD} (Create, Read, Update and Delete) for the model Use Case that satisfies the 
\ac{FeDRE} approach needs. The model Use Case should include a unique Use case id,  name,  description, 
associated feature(s), pre-conditions, post-conditions, and the main success scenario. A Use Case can 
also be related to an actor and may have include and/or extend relationships with other use case(s), 
and alternative scenarios.

\item  \textbf{FR3 - FeDRE Guidelines.} The tool shoul implement the \ac{FeDRE}
guidelines for specifying \ac{SPL} functional requirements.  The Features must be stored  
hierarchically in order to enable \ac{FeDRE} guidelines implementation. This  should be done by 
storing Features as an n-ary tree data structure that  represents the Feature Model. 

\end{itemize}

\section{SPLICE-FeDRE Implementation}
\label{sc:implementation}

The tool \ac{SPLICE} was implemented using the Django Framework and the Python programming language. 
According to \citep{python2014}, “Python is a dynamic object-oriented
programming language that is used in a wide variety of application domains. It has a very clear, readable syntax, 
offers strong support for integration with other languages and tools, comes with extensive standard 
libraries, and can be learned in a few days. Many Python programmers report substantial productivity 
gains and feel the language encourages the development of higher quality, more maintainable code”. 
Other languages used to developed the tool were JavaScript, CSS, HTML, XML, YAML and make.

According to the \ac{SPLICE} developer, this choice was motivated by the unprecedented flexibility that 
the Django \acf{ORM} empowers it users, making the metamodel changes
effortless.
Also, Python is becoming the introductory language for a number of computer science curriculums  
\citep{Sanders2008}. Python is also frequently used on many scientific workflows
\citep{Bui2010} making the project attractive for future data scientists
experiments and for undergraduate projects.
 
The implementation of the new version called SPLICE-FeDRE adopted the same set of programming languages and 
framework for development. In SPLICE-FeDRE, the features are stored hierarchically using a modified preorder 
tree traversal algorithm. We can think of the feature model as 
an n-ary tree of features, where the root node is a special node that represents de product line. This 
tree is traversed using a depth-first search algorithm, then the \ac{FeDRE} flow of activities, tasks and 
steps is executed for each subtree of the root node, one at a time. 

\section{SPLICE-FeDRE in action}
\label{sc:operation}

In order to demonstrate how the tool SPLICE-FeDRE works, this section shows the 
operation of selected features, with a brief description.

\subsection{Feature Specification}

In the home page of SPLICE-FeDRE, seen in \figref{fg:assets}, an assets menu can
be used to manage the assets available.

\begin{figure}[htp]
\begin{center}
  \includegraphics[width=14cm]{chapters/proposed_solution/img/captures/assets.PNG}
  \caption[Assets screen]{Assets screen}
  \label{fg:assets}
\end{center}
\end{figure}

By clicking in Features, an user can have access to a complete \acf{CRUD}, 
which is also available for all the models listed in the assets menu. \figref{fg:add-feature} 
shows part of the form used to add a new Feature while specifying the Features. It implements the functional
requirement \textit{FR1- Feature Specification}. The model Feature includes a unique Feature id, 
name, description, priority (high, medium or low), type (abstract or concrete), variability 
(mandatory, optional, OR ou XOR), binding time (compile, runtime), parent feature, glossary, use case 
diagram, similar feature(s), required feature(s) and excluded feature(s). 

\begin{figure}[htp]
\begin{center}
  \includegraphics[width=14cm]{chapters/proposed_solution/img/captures/add_feature.PNG}
  \caption[Add feature form]{Add feature form}
  \label{fg:add-feature}
\end{center}
\end{figure}

\subsection{FeDRE Guidelines}

The \ac{FeDRE} page is depicted in \figref{fg:start-fedre}. Once the features
specification is finished, the user can start the flow of \ac{FeDRE} guidelines in order to specify the requirements of each subtree of features, 
one by one. It implements the functional requirement \textit{RF3 – FeDRE
Guidelines}.

\begin{figure}[htp]
\begin{center}
  \includegraphics[width=14cm]{chapters/proposed_solution/img/captures/start_fedre.PNG}
  \caption[FeDRE initial screen]{FeDRE initial screen}
  \label{fg:start-fedre}
\end{center}
\end{figure}

For each branch, the user can see a hierarchy of this branch features, and lists of the features that 
must have use cases, the features that may have use cases and the features that should not have use cases. 
Also, it is shown a list of steps to be accomplished before moving to the next
branch, as seen in \figref{fg:branch1}.

\begin{figure}[htp]
\begin{center}
  \includegraphics[width=14cm]{chapters/proposed_solution/img/captures/branch1.PNG}
  \caption[Branch example]{Branch example}
  \label{fg:branch1}
\end{center}
\end{figure}

\subsection{Use Case Specification}

As mentioned above, a complete \ac{CRUD} is also accessible for the model Use
Case.
Implementing the requirement \textit{FR2 - Use Case Specification}, the model
Use Case includes a unique Use case id,  name,  description, associated feature(s), pre-conditions, 
post-conditions, 
and the main success scenario. A Use Case can also be related to an actor and may have include and/or 
extend relationships with other use case(s), and alternative scenarios. See
\figref{fg:add-use-case} below:

\begin{figure}[htp]
\begin{center}
  \includegraphics[width=14cm]{chapters/proposed_solution/img/captures/add_use_case.PNG}
  \caption[Add use case form]{Add use case form}
  \label{fg:add-use-case}
\end{center}
\end{figure}

\section{Summary}
\label{sc:solutionsummary}

In this chapter, it was presented the \acf{FeDRE} approach and the \acf{SPLICE},
a web-based tool for \ac{SPL} lifecycle management, and how it was extended to 
automate the \ac{FeDRE} approach. Next chapter presents an evaluation of SPLICE-FeDRE 
performed during the development of the tool. 



\chapter{SPLICE-FeDRE Evaluation}
\label{ch:caseStudy}

\section{Introduction}
\label{sc:experimentIntroduction}

This chapter describes a survey applied to validate the tool developed for this
work. It is organized as follows: \secref{sc:definition} defines this evaluation; in \secref{sc:researchMethod} the 
data collection model is presented; \secref{sc:datacollection} describes the results and its interpretation; the 
\secref{sc:threats} analyzes the threats to validity of the evaluation; \secref{sc:leassonsLearned} and \secref{sc:expsummary} describe some 
findings and summarize the chapter.

\section{Definition}
\label{sc:definition}

\subsection{Context}
A survey was applied, after the complete implementation of SPLICE-FeDRE, in
order to validate the application developed in regard to its usefulness when it 
comes to handling complexity and scalability problems during requirements specification. 
The survey was applied at Software Engineering Laboratory, in November 2015, and we had 
two participants that were master students, all members of \acf{RiSE} Lab.
\subsection{Research Questions}

In this evaluation the main objective is to analyze the usefulness of the tool for handling 
complexity and scalability problems that may arise during requirements specification phase 
of a \acf{SPL}. In order to evaluate these aspects in our proposal, we 
defined two research questions:
\begin{itemize}
\item \textbf{Is the proposed tool useful for handling compexity during the SPL Requirements Engineering process?}

Rationale: The goal is to verify if the tool helps handling complexity during the \ac{SPL} 
Requirements Engineering process.
\item \textbf{Is the proposed tool useful for handling scalability problems during the SPL Requirements Engineering process?}

Rationale: The goal is to verify if the tool helps handling scalability problems during the \ac{SPL} 
Requirements Engineering process.

\end{itemize}

\section{Data collection}
\label{sc:researchMethod}

\subsection{Survey Design}
The data collection instrument selected in this evaluation is the Expert Survey. A survey is a 
mechanism of data gathering in which participants answers questions or statements previously 
developed and they are probably the most commonly used instrument to gather opinions from experts 
according to \citep{Kitchenham2008}.

This survey design is based on the design proposed by \citep{Kitchenham2008} and it is 
composed of a set of personal questions, closed-ended and open-ended questions related to the research 
questions. Also, to give the respondents the necessary understanding about the application, a training 
was offered to them. The remain of this section contains the overall process applied in this evaluation 
and the methodology used. 
\subsection{Developing the Survey Instrument}
The questionnaire was composed of three personal questions, five closed
questions with justification fields, and three open questions. The closed questions were 
formulated to measure and quality the data, while getting personal feedback. The open questions 
were built to collect the experts’ experiences and their impressions about the tool. The questionnaire 
can be seen in the Appendix 1.

\subsection{Analyzing the data}
In order to collect the data, the experts filled a printed questionnaire. After designing and running 
the survey, the next step was to analyze the collected data. The main analysis procedure was to check 
all responses, tabulate the data, identify the findings and classify the options.

\section{Results}
\label{sc:datacollection}
In this section the analysis of the collected data are presented, discussing the
given answer for each question.

\subsubsection{Respondents experience}
The first three questions were personal questions such as name and experience,
and their answers are summarized in Table \ref{table:expertsselected}.

\begin{table}
\begin{center}
\centering
\small
\tabcolsep=0.11cm
    \begin{tabular}{|l|l|l|l|}
    \hline
    Respondent & Occupation & RE experience & SPL experience           
    \\ \hline 1 & M.Sc student & 4 years & 3 years and 5 months
    \\ \hline 2 & M.Sc student & 6 years & 3 years 
    \\ \hline
    \end{tabular}
        \caption {Experts Selected}
        \label{table:expertsselected}
        \end{center}

\end{table}

\subsubsection {Tool usage difficulties}
Considering the questions \textbf{“Did you have any difficulty during the
execution of any activity in the tool?”} and  \textbf{“Did you have any problems
creating  use cases?”}, none of the respondents reported any difficulty to use
the tool.

\subsubsection{Tool Helpfulness}
In the question \textbf{“Do you think that the proposed tool would aid you
during a SPL Requirements Engineering process? Would you spontaneously use the
tool hereafter?”} all the respondents agreed that the tool would aid them to specify 
requirements and would keep using the tool 
from that moment on. However, one of them stated that the \ac{FeDRE} steps could
be more detailed in the tool.


Considering the question \textbf{“Do you think the proposed tool is useful to
handle the complexity of SPL Requirements Engineering process?”}, both
respondents answered yes, and one of them stated that the tool reduces the effort to specify 
requirements using \ac{FeDRE} guidelines.


All the respondents indicated in the question \textbf{“Do you think the proposed
tool is useful to handle scalability problems during a SPL Requirements
Engineering process?”} that the proposed tool is useful to handle scalability problems that may arise during  
\ac{SPL} Requirements Engineering process.

\subsubsection{Positive points}
We asked the experts the question \textbf{What are the positive points of using the tool?}, and the positive 
points of the tool according to them are:
\begin{itemize}
\item Centralized information;
\item More accuracy when choosing the use cases to be specified;;
\item Time saving.
\end{itemize}

\subsubsection{Negative points}
In Contrast with the previous question, we also asked \textbf{What are the
negative points of using the tool?}.
Only one point was mentioned, as follow:
\begin{itemize}
\item An external tool is needed to draw the use case diagrams.
\end{itemize}

\subsubsection{Suggestions}
Lastly, we asked \textbf{“Please, write down any suggestion you think might be
useful”}. One expert suggested that we implement a use case drawing feature as part of the tool 
to avoid the use of external tools. The another expert suggested a more illustrated interface.

\section{Threats to validity}
\label{sc:threats}
There are some threats to the validity of our study, which were briefly
described and discussed:
\begin{itemize}
\item \textbf{Research questions.} The research questions we defined cannot
provide complete coverage of all the features covered by the tool. We considered just some important points: complexity and 
scalability problems handling.

\item \textbf{Sample size.} The number of respondents is an important detail in
a survey. Due to the limited availability of respondents with a \ac{SPL} background, the evaluation
may contain biases.
A higher number of participants helps  generalizing the results obtained.

\item \textbf{Quality of training.} The quality of the training conducted before
applying the questionnaire may have compromised the correct understanding of the \ac{FeDRE} approach and 
the application evaluated.

\item \textbf{Translation of the answers.} All the responses were written in
portuguese and translated to english by the author. This may have changed the direction of the response.

\end{itemize}

\section{Findings}
\label{sc:leassonsLearned}

\section{Summary}
\label{sc:expsummary}

This chapter presented the definition, planning, analysis and interpretation of a survey 
to evaluate the SPLICE-FeDRE tool. The survey was applied to experts at Software Engineering Laboratory. 
The two participants were members of \ac{RiSE} Lab. After concluding the  
questionnaires, we gathered information that can be used as a guide to improve the tool, and an indicator 
about the actual status of the tool. 
The results of the experiment pointed out that the SPLICE-FeDRE addresses the 
complexity and scalability problems that may arise during \ac{SPL} requirements
engineering phase.
However, some points of improvements were raised, that we plan to fix on future versions. Next 
chapter presents the concluding remarks and future work of this dissertation.



\chapter{Conclusion}
\label{ch:conclusion}

A \acf{SPL} is outlined as a collection of similar software intensive systems
that share a set of common features satisfying the wants of specific customers, market segments 
or mission. Those similar software systems are developed from a set of core assets, comprised of 
documents, specifications, components, and other software artifacts that may be reusable throughout 
the development of each system within the product line
\citep{rafael2013systems}.

Requirements are typical assets in \ac{SPL}. They are specified in reusable models,
in which commonalities and variabilities are documented explicitly. Thus, these 
requirements can be instantiated and adapted to derive the requirements for an 
individual product \citep{cheng2007research}. New products in the SPL will be
much simpler to specify, because the requirements are reused and tailored
\citep{clements2002software}.

\acf{RE} in \ac{SPL} has an additional cost. Many \ac{SPL} requirements are
complex, interlinked, and divided into common, variable and product-specific requirements 
\citep{birk2003report,de2014defining}.  The requirements engineering process
must be tool-supported to handle complexity and the huge volume of elicited requirements
\citep{birk2003report}.

In this work, we proposed a support tool for performing the
specification of the \ac{SPL} requirements in a systematic way through the use of guidelines,  
showing step by step how the specification should be done. In addition, an
initial evaluation was performed in order to point out negative and positive
points of the tool and direct us to future improvements to be done.

\section{Research Contribution}
\label{sc:reserach-contrib}
As a result of the work presented in this dissertation, the following contribution can be highlighted:
\begin{itemize}
\item \textbf{Tool support for a SPL domain requirements specification approach
(FeDRE)} 
We extended the tool \ac{SPLICE}, a \ac{SPL} lifecycle management tool
and automated \acf{FeDRE}, thus improving the automation of Software Product Lines (\ac{SPL}) 
requirements engineering phase.
\end{itemize}

\section{Future Work}
\label{sc:future-work}
An initial version of the tool was developed and evaluated in this work.
However, we are aware that some enhancements and features must be implemented.
Also, some defects must be fixed. 
In this work, we presented a survey. A more
detailed evaluation is needed, for example, a controlled experiment or a case
study with a higher number of respondents in order to provide richer findings for the stakeholders.




\bibliographystyle{natbib}
\addcontentsline{toc}{chapter}{\bibliographytocname}
\bibliography{references}

%Appendix
\clearpage
\addappheadtotoc
\appendix
\appendixpage
\chapter{Evaluation Instruments} 
\label{ap:measurement-instruments}


\def \tick{
$[$\hspace{0.3cm}$]$
}

\def \twooption#1#2{
\tick #1.  \tick #2.
}

\def \threeoption#1#2#3{
\tick #1.\newline
\tick #2.\newline
\tick #3.
}

\def \fouroption#1#2#3#4{
\tick #1.\newline
\tick #2.\newline
\tick #3.\newline
\tick #4.
}

\def \fiveoption#1#2#3#4#5{
\tick #1.\newline
\tick #2.\newline
\tick #3.\newline
\tick #4.\newline
\tick #5.
}
\def \datefield{
%date fied used in time sheets
    /\hspace{0.4cm}/
}

\def \rcolor{
%table row color
    \rowcolor[gray]{0.9}
}

\def \hcolor{
    \rowcolor[gray]{0.7}
}


\section{Form for Expert Survey}
\label{ap:sec:feedback}

\textbf{Name:}
\\
\line(1,0){410}
\\
\\
\textbf{What is your experience with Requirements Specification (in months/years)?}
\\
\line(1,0){410}
\\
\line(1,0){410}
\\
\\
\textbf{What is your experience with Software Product Lines (in months/years)?}
\\
\line(1,0){410}
\\
\line(1,0){410}
\\
\\
\textbf{Did you have any difficulty during the execution of any activity in the tool?}
\\
\twooption{Yes}{No}
\\
\textbf{In case you answered Yes, detail the difficulty encountered:}
\\
\line(1,0){410}
\\
\line(1,0){410}
\\
\line(1,0){410}
\\
\line(1,0){410}
\\
\\
\textbf{Did you have any problems creating use cases?}
\\
\twooption{Yes}{No}
\\
\textbf{In case you answered Yes, describe the problems encountered:}
\\
\line(1,0){410}
\\
\line(1,0){410}
\\
\line(1,0){410}
\\
\line(1,0){410}
\\
\\
\textbf{Do you think that the proposed tool would aid you during a SPL Requirements Engineering process? Would you spontaneously use the tool hereafter?}
\\
\line(1,0){410}
\\
\line(1,0){410}
\\
\line(1,0){410}
\\
\line(1,0){410}
\\
\\
\textbf{Do you think the proposed tool is useful to handle the complexity of SPL Requirements Engineering process?}
\\
\line(1,0){410}
\\
\line(1,0){410}
\\
\line(1,0){410}
\\
\line(1,0){410}
\\
\\
\textbf{Do you think the proposed tool is useful to handle scalability problems during a  SPL Requirements Engineering process?}
\\
\line(1,0){410}
\\
\line(1,0){410}
\\
\line(1,0){410}
\\
\line(1,0){410}
\\
\\
\textbf{What are the positive points of using the tool?}
\\
\line(1,0){410}
\\
\line(1,0){410}
\\
\line(1,0){410}
\\
\line(1,0){410}
\\
\\
\textbf{What are the negative points of using the tool?}
\\
\line(1,0){410}
\\
\line(1,0){410}
\\
\line(1,0){410}
\\
\line(1,0){410}
\\
\\
\textbf{Please, write down any suggestion you think might be useful.}
\\
\line(1,0){410}
\\
\line(1,0){410}
\\
\line(1,0){410}
\\
\line(1,0){410}
\\


\end{document}
